\chapter{植物基因工程}
\section{植物组织培养}
\begin{enumerate}
    \item \textbf{植物组织培养}的定义:在离体条件下,利用人工培养基将植物器官、组织、细胞、原生质体等培养成\uline{完整的植株}的一门生物科学技术。
    \item 植物组织培养的理论依据:\textbf{植物细胞全能性},指植物的每个细胞都包含着该物种的全部遗传信息,从而具备发育成完整植株的遗传能力。在适宜条件下,任何一个细胞都可以发育成一个新个体。
    \item 植物组织培养根据所培养的植物材料不同分为:器官培养、茎尖分生组织培养、愈伤组织培养、细胞培养、原生质体培养。其中,\textbf{愈伤组织培养}是最常见的培养类型。
    \item 
\end{enumerate}
\subsection{植物愈伤组织培养}
\begin{enumerate}
    \item \textbf{愈伤组织}:原指植物在受伤后于其伤口表面形成的一团不定型的薄壁细胞,在植物组织培养中则指在人工培养基上由\textbf{外植体}(从活体植物上取下的无菌细胞、组织、器官等)形成的一团\uline{无序生长的薄壁细胞}。
    \item 植物愈伤组织培养的流程:
    \[
        \text{分生细胞}\overset{\text{分化}}\longrightarrow\text{成熟细胞}\overset{\text{脱分化}}\longrightarrow\text{分生细胞}\overset{\text{再分化}}\longrightarrow\text{愈伤组织}\overset{\text{驯化}}\longrightarrow\text{愈伤组织}
    \]
    \item 从单个细胞或一块外植体形成典型的愈伤组织,大致要经历三个时期:\uline{诱导期}、\uline{分裂期}、\uline{分化期}。
    \item \textbf{人工种子}:通过植物组织培养技术获得具有胚芽、胚根、胚轴等结构的\uline{植物胚状体},并且用适当的方法将胚状体包裹起来,用以代替天然种子进行繁殖的一种结构。
\end{enumerate}
\subsection{原生质体培养}
\begin{enumerate}
    \item \textbf{原生质体}:脱去细胞壁、裸露的细胞。又称原生质球。
    \item 原生质体培养流程:
    \begin{enumerate}
        \item 材料的选择:基因型、材料的类型和生理状态
        \item 材料的预处理:黑暗培养、低温处理、预质壁分离;
        \item 原生质体的分离:机械分离法、酶解分离法;
        \item 原生质体的纯化:离心沉淀法、漂浮法和界面法;
        \item 活力的检测:形态识别(FDA)、活体染色和荧光活体染色;
        \item 原生质体培养;
        \item 原生质体的生长发育和细胞壁的再生;
        \item 植株的再生:愈伤组织或胚状体诱导形成;
    \end{enumerate}
    \item 原生质体纯化的方法:离心沉淀法、漂浮法、不连续梯度法。
    \item 原生质体培养的类型:液体浅层培养、固液双层培养、固体平板培养、琼脂糖珠培养。
\end{enumerate}
\subsection{体细胞杂交}
\begin{enumerate}
    \item \textbf{体细胞杂交}:即\textbf{原生质体融合},指利用适当的物理或化学方法,将两个不同亲本的原生质体融合在一起,利用适宜的培养方法,使融合原生质体再生出杂种植株。
    \item 体细胞杂交与有性杂交的异同:
    \begin{table}[h]
        \centering
        \begin{tabular}{ccc}
            \toprule
            &\textbf{体细胞杂交}&\textbf{有性杂交}\\
            \midrule
            \textbf{时间}&无季节限制&花期限制\\
            \textbf{亲缘关系}&一定程度克服不亲和性&受亲和性影响\\
            \textbf{结果}&核质重组,倍性增加&几乎无细胞质重组,倍性不变\\
            \bottomrule                        
        \end{tabular}
    \end{table}
    \item 体细胞杂交的步骤:
    \begin{enumerate}
        \item 用\uline{纤维素酶}、\uline{果胶酶}处理细胞壁;
        \item 诱导融合;
        \item 细胞壁再生,脱分化成\uline{愈伤组织},之后与植物组织培养相同。
    \end{enumerate}
    \item 诱导融合的方法:
    \begin{enumerate}
        \item 化学方法:聚乙二醇(PEG)、硝酸钠、高pH+高浓度钙离子;
        \item 物理方法:电脉冲、飞秒激光诱导融合、电融合芯片、空间细胞融合;
        \item 生物方法:仙台病毒。
    \end{enumerate}
\end{enumerate}
\subsection{单倍体细胞培养}
\begin{enumerate}
    \item \textbf{单倍体植物}:染色体数目只有孢子体一半的配子体叫做单倍体植物。
    \item 单倍体植物的特点:不孕性。染色体加倍后可获得\textbf{加倍单倍体植株(Doubled Haploid,DH植株)}。
    \item 诱导单倍体植物的方法:
    \begin{enumerate}
        \item 人工诱导孤雌生殖;
        \item 染色体消除:小麦和玉米杂交后,可通过杂合子中玉米染色体的自发消除形成小麦单倍体;
        \item 半配合法:植物授粉后,雄配子核进入卵细胞,但并不与卵细胞融合,而是彼此独立分裂,最后形成镶嵌有来源于父本和母本组织的异质胚;
        \item 未授粉子房或胚珠培养;
        \item 花药培养和小孢子培养。
    \end{enumerate}
    \item 单倍体植物在育种上的应用:
    \begin{enumerate}
        \item 加快常规育种的速度;
        \item 在加倍单倍体中隐性性状得以表现;
        \item 取代自交快速获得自交系;
        \item 诱导超雄植株和全雄性杂种;
        \item 单倍体细胞突变体筛选;
        \item 利用DH群体绘制遗传图谱。
    \end{enumerate}
    \item 单倍体植株的加倍处理方法:
    \begin{enumerate}
        \item 自发加倍:核内有丝分裂、花粉核的融合;
        \item 人工加倍:秋水仙素处理。
    \end{enumerate}
\end{enumerate}
\subsection{无毒苗的培育}
\begin{enumerate}
    \item \textbf{热处理脱毒}:利用某些病毒受热以后的不稳定性,而使病毒钝化、失活,如播种前“烫种”。
    \item \textbf{茎尖培养脱毒}:病毒在植物体内的分布是不均匀的,在受感染的植物中顶端分生组织通常不含或仅含低浓度的病毒。茎尖快速生长,病毒传播通过胞间连丝,较慢。茎尖部分新陈代谢活跃,呼吸旺盛,抑制病毒生长。
    \item \textbf{微体嫁接脱毒}:利用组织培养与嫁接方法相结合来获得无病毒植株。
    \item \textbf{抗病毒药剂脱毒}:如抗病毒醚等。
\end{enumerate}
\subsection{植物种质资源保存}
\begin{enumerate}
    \item \textbf{种质保存}:利用天然或人工创造适宜环境保存种质资源,使个体所含有的遗传物质保存其完整性,有高的活力,能通过繁殖将遗传特性传递下去。
    \item 种质保存的分类:
    \begin{enumerate}
        \item \textbf{低温保存}:1-9$^\circ$C;
        \item \textbf{超低温保存}:$<-80^\circ$C,使用液氮。
    \end{enumerate}
    \item 低温保存的方法与技术:
    \begin{enumerate}
        \item 控制培养环境温度;
        \item 改良培养基中特定营养物质浓度(饥饿法);
        \item 提高培养基渗透压;
        \item 培养基中加入生长抑制剂(ABA);
        \item 低压保存(低气压、低氧压);
        \item 干燥保存(适当干燥失水,减缓生长)。
    \end{enumerate}
    \item 超低温保存的方法与技术:
    \begin{enumerate}
        \item 常规超低温保存:通过调控预培养,冷冻保护剂处理时间,降温速度等关键环节,创造合适的脱水程度,实现超低温保存。
        \item 玻璃化法超低温保存:生物材料经高浓度玻璃化保护剂处理后,快速投入液氮保存,使保护剂和细胞内水分来不及形成冰晶。不会造成机械损伤或溶液效应,伤害组织和细胞。
        \item 包埋脱水法超低温保存:将包含有样品的褐藻酸钠溶液滴向高钙溶液,生成褐藻酸钙颗粒,同时辅以高浓度蔗糖预处理样品,使样品获得高抗冻力和抗脱水力,结合适当的降温和脱水方式,液氮下保存。
        \item 干燥冷冻法超低温保存:快速冷冻保存法。将植物材料干燥处理,适度脱水,控制脱水速度。
    \end{enumerate}
\end{enumerate}

\section{植物遗传转化技术}
\subsection{病毒介导的植物转化}
\begin{enumerate}
    \item 原理:将外源目的基因插入到\uline{双链DNA病毒基因组}上,重组分子在体外包装成有感染力的病毒颗粒,就可高效转染植物原生质体,进而通过原生质体培养再生为整株植物。
    \item 成熟的两种病毒转化载体:\uline{花椰菜花斑病毒(CaMV)}、\uline{番茄金花叶病毒(TGMV)}。
    \item 缺陷:
    \begin{enumerate}
        \item 病毒DNA容量有限,可感染的寄主范围有限;
        \item 使寄主染病,产量下降;
        \item 缺乏有效选择标记;
        \item 不能插入寄主植物基因组,无法稳定遗传。
    \end{enumerate}
\end{enumerate}
\subsection{基因的直接转化}
\begin{enumerate}
    \item \textbf{基因枪}:利用火药爆炸或高压气体加速,将包裹了带目的基因的DNA溶液的高速微弹 (直径4μm的钨或金颗粒)直接送入完整的植物组织和细胞中。
    \item \textbf{花粉管法}:在授粉后向子房注射含目的基因的DNA溶液,利用花粉管通道将外源DNA导入受精卵细胞,并进一步地被整合到受体细胞的基因组中,随着受精卵的发育而成为带转基因的新个体。
    \item \textbf{显微注射}:利用显微注射仪等,通过机械的方法将外源基因或DNA直接注人细胞核或细胞质。
    \item \textbf{电击注射}:电脉冲能改变细胞膜的透性。通过高压电脉冲的电激穿孔作用把外源DNA引入植物原生质体的方法。
    \item \textbf{PEG介导的基因转化}:利用PEG(聚乙二醇)和多聚L-鸟氨酸、磷酸钙及高pH值条件下诱导原生质体摄取外源DNA分子。
\end{enumerate}
\subsection{农杆菌转化}
\begin{enumerate}
    \item \textbf{Ti质粒}:大肠-农杆菌穿梭质粒,存在于根瘤农杆菌,诱导植物产生冠瘿瘤。
    \item \textbf{Ri质粒}:存在于发根土壤农杆菌,控制不定根的形成。
    \item \textbf{T-DNA区(transferred-DNA regions)}:Ti或Ri质粒中可从农杆菌中转移并稳定整合到植物核基因组中的DNA序列,使插入其中的外源基因在植物体内得以表达。它的两侧是\uline{两个25bp重复边界序列}。农杆菌中的T-DNA区含有控制合成生长素和细胞分裂素的基因,从而导致肿瘤生长。
    \item \textbf{Vir区(virulence region)}:T-DNA以外涉及诱发肿瘤的区域, 该区基因能激活T-DNA转移,使农杆菌表现出毒性,故称之为毒区。
    \item \textbf{Con区(regions encoding conjugations)}:接合转移编码区,带有与细菌间接合转移的有关基因\emph{tra},受冠瘿碱激活,调控Ti质粒在农杆菌之间的转移。
    \item \textbf{Ori区(origin of replication)}:复制起始区,该区基因调控Ti质粒的自我复制。
    \item 农杆菌\uline{侵染植物伤口}时,通过一系列过程进入植物细胞并\uline{将其质粒上的T-DNA插入植物基因组中}。Ti质粒转化的步骤:
    \begin{enumerate}
        \item 根癌农杆菌对植物细胞的识别和附着;
        \item 根癌农杆菌对植物信号物质的感受;
        \item 根癌农杆菌Ti质粒上的\emph{vir}基因以及染色体上操纵子的活化;
        \item T-DNA复合体的产生;
        \item T-DNA复合体的转运;
        \item T-DNA整合到植物基因组中。
    \end{enumerate}
    \item 农杆菌可通过感染\uline{花、愈伤组织、茎尖器官}介导遗传转化。
    \item T-DNA和\emph{vir}基因可以被安在不同的质粒上,从而降低分子克隆的难度。
    \item 核酸酶\textbf{VirD2}在边界序列处切除T-DNA,从而从质粒上分离单链T-DNA并与其保持共价连接辅助其入核。
    \item \textbf{VirE2}保护T-DNA链使其免受核酸酶的降解。
\end{enumerate}

\section{植物基因的克隆原理}
\subsection{植物基因的结构、特点及类型}
\begin{enumerate}
    \item 植物基因的结构:5'上游区、5'非翻译区、编码区、3'非翻译区。
    \begin{enumerate}
        \item 5'上游区:转录起始位点(CTC$\overset{+1}{\text{A}}$TCA)、TATA box、CAAT box(增强转录)、顺式作用元件;
        \item 5'非翻译区:转录起始位点至翻译起始密码子之间区域;
        \item 编码区:起始密码子至终止密码子;
        \item 3'非翻译区:终止密码子后。
    \end{enumerate}
    \item 植物基因组的类型:核基因组、叶绿体基因组、线粒体基因组。
    \item 载体的分类:
    \begin{enumerate}
        \item 按用途分:克隆载体(pUC系列、pMD19-T)、表达载体(酵母表达质粒);
        \item 按来源分:质粒载体、噬菌体载体、黏粒。
    \end{enumerate}
    \item 质粒的存在形式:\uline{超螺旋}、\uline{开环双螺旋}、\uline{线状双螺旋}。
    \item 噬菌体的生长周期类型:\uline{溶菌型}、\uline{溶原型}。
    \item 植物基因克隆的工具酶:\uline{限制酶}、\uline{DNA连接酶}、\uline{DNA聚合酶}、\uline{修饰性工具酶}。
\end{enumerate}

\section{基因组文库的构建}
\begin{enumerate}
    \item 基因组DNA的制备:制备的DNA分子量越大,经切割处理后样品中含有不规则末端的DNA片段比率就越低,重组率和完备性也就越高。
    \item 基因组DNA的切割:超声波处理、限制性内切酶部分酶切。
    \item 载体的选择:最大装载量YAC>BAC>λ-DNA>黏粒。
    \item 后续分析和应用:从文库中筛选目的基因、全基因组测序。
\end{enumerate}

\section{植物基因的克隆原理}
\begin{enumerate}
    \item \textbf{基因克隆}:利用体外重组技术,将特定的基因和其他DNA顺序插入到载体分子中。
    \item 获得目的基因的手段:
    \begin{enumerate}
        \item 化学法直接合成;
        \item 从基因组文库中钓取;
        \item 从cDNA文库中钓取;
        \item 通过PCR直接扩增。
    \end{enumerate}
    \item 重组载体转化宿主细胞的方法:钙离子转化法、电转化法。
    \item 重组克隆的筛选与鉴定:
    \begin{enumerate}
        \item 抗性筛选;
        \item 蓝白斑筛选;
        \item PCR筛选和限制酶酶切;
        \item 核酸分子杂交;
        \item 免疫学筛选。
    \end{enumerate}
    \item DNA测序方法:
    \begin{enumerate}
        \item 双脱氧链终止法;
        \item 化学降解法;
        \item 自动化测序;
        \item 基因芯片测序。
    \end{enumerate}
    \item 外源基因的表达:
    \begin{enumerate}
        \item 瞬时表达:瞬时表达是未整合的外源基因的表达。
        \item 稳定表达:外源基因整合到和基因组DNA分子上,并能稳定地遗传外源基因的表达,分为组成型表达和诱导型表达。
    \end{enumerate}
\end{enumerate}

\section{转基因与生物安全}
\begin{enumerate}
    \item 转基因你和我新品种培育的主要过程:$\text{基因分离}\longrightarrow\text{基因克隆}\longrightarrow\text{基因设计}\longrightarrow\text{遗传转化}\longrightarrow\text{转化体筛选分析}\longrightarrow\text{回交转育}$
    \item 育种技术发展的阶段:$\text{驯化育种}\longrightarrow\text{选择育种}\longrightarrow\text{杂交育种}\longrightarrow\text{诱变育种}\longrightarrow\text{分子育种}$
    \item Bt蛋白利用鳞翅目体内的碱性蛋白酶、肠壁表皮细胞上存在的Bt蛋白受体杀虫。人类胃肠道不存在相应碱性蛋白酶和Bt蛋白受体。
\end{enumerate}