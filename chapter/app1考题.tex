\chapter{往年考题}

\section{2020年考题}

\begin{enumerate}
    \item 植物细胞壁的主要化学成分是什么?
    \item 植物线粒体内的NAD$^+$苹果酸酶的作用是什么?
    \item 简述植物和微生物体内乙醛酸循环的主要酶促反应。
    \item 简述植物韧皮部同化物运输的压力流学说。
    \item 植物气孔运动的直接动力是保卫细胞内外的水势差,这种说法对吗?请简述理由。
    \item 简述双受精现象。
    \item 什么是长日植物、短日植物、日中性植物?
    \item 生长素合成的前体物质是什么?
    \item 成熟的雌配子体——胚囊内有几个细胞?分别是什么?
    \item 为什么植物体矿质元素的跨膜运输方式叫做次级主动运输?
    \item 简述植物科学发展的历史(主要时期及相应特点)。
    \item 如果你得到一段水稻的基因序列,请设计一条探究该基因功能的思路。
    \item 列举两种重要的药用植物、相应的有效物质及用途。
    \item 叙述根瘤农杆菌介导植物遗传转化的原理,并介绍T-DNA及二元载体系统(binary vector)。
    \item 植物组织培养的理论依据是什么?
    \item 叙述“碳掩埋”的概念。
\end{enumerate}
