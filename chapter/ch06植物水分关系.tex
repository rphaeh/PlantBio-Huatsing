\chapter{植物水分关系}
\section{植物细胞水分关系}
\subsection{水势的概念}
\begin{enumerate}
    \item 在植物生理学中,\textbf{水势}($\Psi$)是\uline{每偏摩尔体积水的化学势差(水的化学势差指特定体} \uline{系的水与纯自由水之间的化学势差)}:
    \[
        \text{水势}=\frac{\text{水的化学势差}}{\text{水的偏摩尔体积}(V_w)}
    \]
    \item 物理化学中,体系中某组分的\textbf{化学势}指:\uline{等温等压条件下,在无限大的体系中,加} \uline{入1mol该物质时引起体系自由能的改变量}。化学势的绝对值是不能测定的,通常以给定状态和人为规定的标准状态的差值来表示。
    \item 水势的概念可以理解为体系中的水与纯水之间每单位体积的自由能差。
    \item \textcolor{red}{体系中水势的组分:}
    \[
        \Psi_w=\Psi_s+\Psi_m+\Psi_p+\Psi_g
    \]
    \begin{enumerate}
        \item $\Psi_s$(\textbf{溶质势}):由于\uline{溶质颗粒的存在}而引起体系水势降低的数值,也称\textbf{渗透势}。计算公式:
        \[
            \Psi_s=-iCRT
        \]
        \begin{itemize}
            \item $R$:气体常数(8.31$\text{J}\cdot\text{mol}^{-1}\cdot\text{K}^{-1}$);
            \item $T$:绝对温度(K);
            \item $C$:摩尔浓度($\text{mol}\cdot\text{L}^{-1}$);
            \item $i$:溶解系数。
        \end{itemize}
        \item $\Psi_p$(\textbf{压力势}):由于溶液静水压(膨压)的存在而使体系水势改变的数值。膨压一般为正;但植物细胞中溶液的静水压也可以为负值,如在根、茎部的木质部导管中。同一大气压下,讨论两个开放体系间的水势差时,$\Psi_p$可以忽略不计。
        \item $\Psi_g$(\textbf{重力势}):重力作用使水向下移动,使处于较高位置的水比较低位置的水有高的水势。当体系中的两个区域高度相差不大时,重力势可忽略。
        \item $\Psi_m$(\textbf{衬质势}):亲水的衬质与水的相互作用使水势降低,把这种衬质对水势产生的影响称为衬质势。如干燥的木材、种子等具有很低的$\Psi_m$,可达-300MPa,因此有很强的吸水能力。
    \end{enumerate}
    \item 纯水的水势为0,化学势也为0。
    \item 植物细胞的水势:
    \begin{enumerate}
        \item 无液泡细胞:$\Psi_w=\Psi_m$;
        \item 有液泡细胞:$\Psi_w=\Psi_m+\Psi_p$
    \end{enumerate}
\end{enumerate}

\subsection{水分运动的形式}
\begin{enumerate}
    \item 水分运动的动力是水势差。水分的移动是顺着能量梯度的方向进行的,\uline{水分总是从水势高处移向水势低处,没有任何例外!}
    \item 水分运动的形式:
    \begin{enumerate}
        \item \textbf{扩散}:浓度梯度所推动的物质运动。物质分子(气体、水、溶质分子等)从高浓度区域向低浓度转移,直至均匀分布的现象。叶片气孔蒸腾作用是由植物充满水汽的气孔下腔向水分亏缺的大气扩散水汽的过程;CO$_2$相反。
        \item \textbf{集流}:压力梯度所推动的物质运动。液体中成群的原子或分子在压力梯度下共同移动的现象,与物质浓度无关。水在导管或筛管中的移动是植物体内主要的“集流”水分运动。
        \item \textbf{渗透}:渗透势梯度所推动的水分运动。溶液中的溶剂分子通过半透膜扩散的现象。
    \end{enumerate}
    \item 植物细胞间也可发生水分运动,相邻两个细胞之间水分移动的方向,取决于两细胞间的水势差。
\end{enumerate}
\subsection{细胞吸水的形式}
\begin{enumerate}
    \item 无论哪种吸水方式,推动力都是\uline{水势差}。
    \item 细胞吸水的形式:
    \begin{enumerate}
        \item \textbf{吸胀吸水}:依赖于衬质势的吸水。发生的条件:\uline{有亲水衬质的存在}、\uline{有衬质导致的水势梯度存在}。
        \item \textbf{降压吸水}:因压力势降低而引起的细胞吸水(集流)。细胞压力势经常为正值,很少情况下为负值。\uline{导管}中的压力势为负值,为降压吸水。
        \item \textbf{渗透吸水}:依赖于渗透压的吸水,为植物细胞吸水或脱水的主要动力。高渗环境下,植物细胞脱水时发生质壁分离现象。
    \end{enumerate}
    \item \textbf{水通道蛋白}或\textbf{水孔蛋白}是分子量为25-30KDa、选择性高效转运水分子的膜蛋白。植物细胞膜水通道蛋白能\uline{提高水分传输速率},但\uline{不改变由水势差决定的水分运输方向}。
    \item 质壁分离过程中的水势变化 :
    \begin{enumerate}
        \item 初始质壁分离:$\Psi_p=0$,$\Psi_w=\Psi_s$;
        \item 充分饱和时:$\Psi_w=0$,$\Psi_p=-\Psi_s<0$。
    \end{enumerate}
\end{enumerate}
\subsection{植物组织水势的测定}
\begin{enumerate}
    \item 干湿球温度计:当水溶液的水势降低时水的蒸气压就会降低,表面蒸发量下降,温度升高。相反,较高的水势伴随表面水分蒸发增加,导致表面温度降低。该方法测定慢,但是准确。
    \item 压力室法:认为\uline{木质部溶液的水势}是和植物组织的水势相近的,因此只要测量出木质部溶液的水势,即可得到植物组织的水势。在植物茎切割前,水柱是连续的,处于负压状态;切割后,水柱在负压作用于缩进植物茎;在压力室正压力的作用下,水柱回到切割表面。可室外快速测定,但数据的准确度相对较低。
    \item 冰点下降法:当溶液中溶质浓度上升时,溶液的冰点会下降。测定溶液的冰点温度,可以计算得出溶质势。
    \item 压力探针法:(单细胞膨压测量)玻璃毛细管刺入细胞时,细胞液由于膨压而进入毛细管。推动活塞,使硅油/细胞液界面返回细胞中。这时压力传感器所测出的平衡压力即为膨压值。
\end{enumerate}

\section{植物整体水分平衡}
\subsection{根对水分的吸收}
\begin{enumerate}
    \item 土壤中水分的状态:在大多数情况下土壤中溶质的含量很低,溶质势高,一般 -0.02 MPa,可以忽略不记;盐碱地或干旱条件下,溶质势降低,水势也随之降低。在一般的湿润土壤的条件下,土壤的水势接近纯水的水势。在大多数情况下,土壤的水势是高于植物根水势的。
    \item 按能否被植物利用分为可利用水、不可利用水。区分的指标:\textbf{永久萎蔫系数}。
    \item \textbf{萎蔫}:当植物体内水分亏缺时,植物细胞膨压下降,叶片、幼茎下垂的现象。
    \begin{enumerate}
        \item 暂时萎蔫:当蒸腾速率降低后,萎蔫植物可恢复正常;
        \item 永久萎蔫:蒸腾降低后,萎蔫植物仍不能恢复正常。
    \end{enumerate}
    \item \textbf{永久萎蔫系数}:植物发生永久萎蔫时,土壤中尚存留的水分占土壤干重的百分率,称为永久萎蔫系数,土壤水势称为\textbf{永久萎蔫点}。永久萎蔫系数以上的水为可利用水,以下的水为不可利用水或无效水。
    \item 根吸水的部位:主要在\uline{根顶端}部分,特别是\uline{根毛区(成熟区)}和\uline{伸长区}。
    \item 根毛区吸水能力最强,原因:
    \begin{enumerate}
        \item 根毛增大了吸水面积;
        \item 根毛外壁,果胶质覆盖,亲水性好;
        \item 根毛区输导组织发达,阻力小,水分移动速度快。  
    \end{enumerate}
    \item 水分进入植物的过程:
    \begin{enumerate}
        \item 质外体途径:水分由细胞壁、细胞间隙、胞间层以及导管的空腔组成的质外体部分的移动过程。
        \item 共质体途径:水分依次从一个细胞的细胞质经过胞间连丝进入另一个细胞的胞质的移动过程。
        \item 共质体、质外体交替途径。
    \end{enumerate}
    \item 经表皮和皮层时,水分可以选择质外体还是共质体途径。但无论如何,水分最终必须跨越内皮层的“凯氏带”细胞(纵向壁和横向壁上形成的一条细的木栓质或类木质素的沉积带,细胞壁不透水),即经共质体途径进入木质部导管。所以,理论上说,进入木质部导管的水分均是经共质体过滤的水分。
\end{enumerate}

\subsection{水在植物体内的运输}
\begin{enumerate}
    \item \uline{木质部}是植物体内进行水分运输的主要途径。木质部管部分子:导管和管胞。
    \item 植物\uline{输导组织}有2类结构:导管和管胞,均为死细胞结构,\uline{细胞壁强烈木质化加厚}。许多个\textbf{导管分子}以细胞的两端连接起来就形成了导管。管胞是单个细胞,\uline{端壁没有穿孔},上下连接的管胞靠侧壁上的纹孔传导水分,导水效率低。
    \item 导管:被子植物和一些裸子植物有。管胞:被子植物和裸子植物都有。
    \item 木质部水分向上运输的机制:主动吸水、被动吸水。
    \item \textbf{主动吸水}:指由于\uline{根系的生理活动}引起的吸水过程,与地上部分的蒸腾无关。主动吸水的推力很弱,只有在土壤水分供应充足时才有可能发生。
    \item 主动吸水的证据:吐水、伤流;动力:根压。
    \item \textbf{根压}:由于根系的生理活动产生的促进液流从根部上升的压力。
    \item \textbf{伤流}:土壤水分充足时,一年生植物幼苗在茎基部切断,可以看见切面木质部有液滴流出,持续数小时或更久,这种现象称为伤流。土壤水分充足时,春天修剪后可以观察到多年生树木的伤流。
    \item \textbf{吐水}:在温暖湿润的傍晚或清晨,常可看到植株叶片尖端或边缘排出水滴,这种现象称为吐水。吐水现象在禾本科植物最为常见。
    \item 根压产生的原因:
    \begin{itemize}
        \item 内皮层存在凯氏带;
        \item 根系细胞主动吸收土壤溶液中的离子;
        \item 离子释放进入导管;
        \item 根木质部导管水势下降,根皮层质外体水势高,内皮层内外产生水势差,推动水分吸收,由此产生静水压——根压。
    \end{itemize}
    \item \textbf{被动吸水}:植物根系以\uline{蒸腾拉力}为动力的吸水过程。
    \item \textbf{蒸腾拉力}:指因叶片蒸腾作用而产生的使导管中水分上升的力量。
    \item 解释蒸腾拉力的\textbf{内聚力-张力学说}:
    \begin{itemize}
        \item 内容:植物在顶部的蒸腾作用会产生很大的\uline{负静水压(张力)},这个负压可以将导管的水柱向上拖动形成水分的向上运输。水分子间的\uline{内聚力}保证了木质部水柱的连续性。
        \item 证据:
        \begin{enumerate}
            \item 木质部导管承受着负压(压力室法测定:树木茎干中的负压达到-1.2至-3.5 MPa);
            \item 负压下水柱的完整性:水分子间具有很强的内聚力,水柱可以抵抗-30 MPa的张力,远远高于植物木质部可能的负压;
            \item 导管中的“气穴”栓塞的克服。
        \end{enumerate}
        \item 排除“气穴”的方法:空气难以通过纹孔膜,把气泡阻挡在导管分子或管胞的两端,而水可以通过侧壁的纹孔对进入相邻的导管或管胞。这就维持了水柱的连续性。气泡在蒸腾很弱时会消失。
    \end{itemize}
\end{enumerate}

\section{蒸腾作用}
\begin{enumerate}
    \item \textbf{蒸腾作用}:水从植物地上部分以水蒸汽状态向外界散失的过程称为蒸腾作用。
    \item 蒸腾作用方式:
    \begin{enumerate}
        \item 皮孔蒸腾:茎;
        \item 叶片蒸腾:角质、气孔。
    \end{enumerate}
    \item 不同植物主要蒸腾作用方式不同:
    \begin{enumerate}
        \item 幼小植株:整个植物体(地上部)蒸腾;
        \item 长成植物:以及气孔蒸腾为主;
        \item 水生植物:以角质蒸腾为主;
    \end{enumerate}
    \item 蒸腾作用的指标:
    \begin{enumerate}
        \item \textbf{蒸腾速率}:植物在单倍时间内单倍面积通过蒸腾作用所散失的水量,也叫\textbf{蒸腾强度}(g$\cdot$m$^{-2}\cdot$h$^{-1}$)。
        \item \textbf{蒸腾系数}:植物光合作用固定每摩尔的CO$_2$所需蒸腾散失的水的量。
    \end{enumerate}
    \item 蒸腾速率的测定:
    \begin{enumerate}
        \item 称重法:离体器官快速称重,测植株重量变化。
        \item 气量计:测定相对湿度的短期变化。
        \item 红外线分析仪:测定温度变化。
    \end{enumerate}
    \item 气孔蒸腾是陆生植物在进化过程中形成的解决CO$_2$吸收和水分蒸发之间矛盾的一个有效机制。
    \item 不同植物叶片的气孔分布:
    \begin{enumerate}
        \item 双子叶植物:主要在下表皮;
        \item 单子叶植物:上下表皮;
        \item 木本植物:只分布在下表皮;
        \item 水生植物:只分布在上表皮。
    \end{enumerate}
    \item \textbf{气孔复合体}:保卫细胞、副卫细胞或邻近细胞以及保卫细胞中的小孔。
    \item 保卫细胞的类型:
    \begin{enumerate}
        \item \textbf{肾形保卫细胞}(有或没有副卫细胞)的\uline{内壁(靠气孔一侧)厚而外壁薄},\uline{微纤丝从气孔呈扇形辐射排列}。当保卫细胞吸水膨胀时,较薄的外壁易于伸长,向外扩展,但微纤丝难以伸长,于是将力量作用于内壁,把内壁拉过来,于是气孔张开。
        \item \textbf{哑铃型保卫细胞}(有副卫细胞)\uline{中间部分的胞壁厚,两头薄},\uline{微纤丝径向排列}。当保卫细胞吸水膨胀时,微纤丝限制两端胞壁纵向伸长,而改为横向膨大,这样就将两个保卫细胞的中部推开,于是气孔张开。
    \end{enumerate}
    \item 气孔运动的动力:保卫细胞的吸水膨胀或失水收缩。
    \item 气孔开放时间: 白天开放晚上关闭;CAM(景天酸代谢)途径植物白天关闭,晚上开放。
    \item 外界环境因素对气孔运动的影响:
    \begin{enumerate}
        \item 光照:光照下张开,黑暗下关闭;
        \item CO$_2$:低浓度CO$_2$导致气孔张开;
        \item 空气湿度:水分亏缺时气孔关闭;
        \item 温度:高温导致气孔关闭或开放;
        \item 风:风使气孔关闭,间接作用。
    \end{enumerate}
    \item 气孔运动的内在调节机制:
    \begin{enumerate}
        \item 气孔运动的渗透调节机理:通过改变渗透势和膨压实现,渗透调节机制是基础调节机制。
        \item 在蓝光照射下气孔扩大:依赖于保卫细胞的光合作用;被蓝光和红光信号系统所推动。  
        \item ABA对气孔运动的调节:水分亏缺诱导保卫细胞内外ABA浓度升高,引起气孔关闭。
    \end{enumerate}
    \item 渗透调节机理的三种学说:
    \begin{enumerate}
        \item 质子泵假说:光激活保卫细胞策膜H$^+$-ATPase,$H^+$泵出,保卫细胞pH升高,内向$K^+$通道打开,K$^+$进入保卫细胞(伴随Cl$^-$进入),水势降低,水分进入,气孔张开。(ABA正好相反)
        \item 苹果酸代谢假说:光导致保卫细胞内CO$_2$消耗,pH升高,活化PEP羧化酶,促进淀粉降解为PEP,PEP与CO$_2$生成草酰乙酸,被NADPH还原为苹果酸,水势降低,水分进入,气孔张开。
        \item 蔗糖-淀粉假说:淀粉转化为蔗糖导致水势降低,水分进入,气孔张开。
    \end{enumerate}
    \item 气孔的运动可能有不同的渗透调节阶段:
    \begin{enumerate}
        \item 日出时光照引起的气孔张开阶段,保卫细胞对钾离子的吸收是主要的渗透调节机制;
        \item 日出后保卫细胞的蔗糖浓度逐渐提高,成为主要的渗透物质。
    \end{enumerate}
    \item 蓝光诱导气孔开放的机制:
    \begin{itemize}
        \item \uline{蓝光紫外光受体向光素}自我磷酸化而激活,诱导保卫细胞内\uline{钙浓度变化},激活H$^+$-ATPase;
        \item 红光受体光敏色素和蓝光紫外光受体隐花色素也参与了气孔开放调节。
    \end{itemize}
    \item 蒸腾作用的意义:
    \begin{enumerate}
        \item 蒸腾拉力是植物吸收水分与传导水分的动力;
        \item 调节植物体温;
        \item 促进木质部内溶液中物质的运输;
        \item 蒸腾作用的正常进行有利于CO$_2$吸收和同化。            
    \end{enumerate}
\end{enumerate}