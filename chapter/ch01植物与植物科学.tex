\chapter{植物科学导论}
\section{植物和植物科学}
\begin{enumerate}
    \item 植物的基本特征:
    \begin{enumerate}
        \item \textbf{自养}生物,具有叶绿体;
        \item 具有\textbf{细胞壁}(纤维素的网状结构);
        \item 具有\textbf{固着生活方式};
        \item 具有\textbf{永久分生组织},不断生长、分化。
    \end{enumerate}
    \item 植物科学的定义:植物科学是以整个植物界为研究对象的科学。包括植物各类群的形态结构、遗传特性、生命活动、发育规律、分类方法以及植物和外界环境间多种多样关系。
    \item 研究植物科学的意义:
    \begin{enumerate}
        \item 帮助保护濒临灭绝的植物和受威胁的环境;
        \item 进一步了解自然世界;
        \item 更好地利用植物为我们提供食物,药物和能源的能力。
    \end{enumerate}
    \item 从对植物的研究中得到的科学发现:
    \begin{enumerate}
        \item Robert Hooke最先从对软木塞的观察中\uline{发现了细胞};
        \item \uline{最早发现的病毒}是烟草花叶病毒;
        \item Mendel对豌豆性状的研究揭示了\uline{遗传规律}。
    \end{enumerate}
    \item \textcolor{red}{植物科学发展的时期:}
    \begin{enumerate}
        \item \textbf{描述植物学时期}:18世纪以前
        \begin{itemize}
            \item 发展动力:人类生活、生产及生存;
            \item 内容:主要是认识和描述植物形态、习性、物候、用途等,积累资料,发展栽培植物;
            \item 方法:描述和比较(具思辩性)
        \end{itemize}
        \item \textbf{实验植物学时期}:18世纪至20世纪初
        \begin{itemize}
            \item 背景:显微镜使用,数、理、化等自然科学的成就及发展,农业及人们生活的需要等;
            \item 研究方法:实验——解剖、分类、生理等
        \end{itemize}
        \item \textbf{现代植物学时期}:20世纪初至今
        \begin{itemize}
            \item 背景:分子生物学及其技术的发展对当代植物科学影响巨大。特别是确认DNA为遗传的物质基础,并阐明了DNA的双螺旋结构之后,分子遗传学带动了植物学和整个生物学的迅速发展。
            \item 研究方法:应用先进技术从分子水平上去研究生命现象。
        \end{itemize}
    \end{enumerate}
    \item \textcolor{red}{双子叶植物和单子叶植物的特征}:
    \begin{table}[h]
        \begin{tabular}{C{2cm}C{4cm}C{4cm}}
            \toprule
            &\textbf{单子叶植物}&\textbf{双子叶植物}\\
            \midrule
            \textbf{子叶数}&一片&两片\\
            \textbf{种子营养储存}&有胚乳&胚乳退化,子叶储藏营养\\
            \textbf{根系}&须根系,主根不发达&直系,主根发达\\
            \textbf{内皮层}&马蹄形加厚&有凯氏带\\
            \textbf{维管柱类型}&有限外韧形&无限外韧型\\
            \textbf{茎}&少有周皮&木质茎最外层为周皮\\
            \textbf{叶表皮}&细胞排列规则,上表皮含泡状细胞&细胞排列紧密不规则,有复表皮\\
            \textbf{叶脉}&大部分为平行脉序&大部分为网状脉\\
            \textbf{叶肉}&栅栏组织和海绵组织无明显分化&分为栅栏组织和海绵组织\\
            \textbf{举例}&水稻、小麦、玉米、高粱、黍&黄豆、花生\\
            \bottomrule
        \end{tabular}
    \end{table}
    \item \textcolor{red}{目前全世界已知植物约\uline{38万种},共\uline{9大门},其中被子植物\uline{226,000种}。}
\end{enumerate}

\section{植物的作用}
\begin{enumerate}
    \item 推动地球和生物界的发展和进化。
    \begin{itemize}
        \item \uline{38亿年前}出现生命,\uline{35亿年前}出现产氧光合作用。已知最早进行产氧光合作用的生物是\textbf{蓝细菌}。\textcolor{red}{早期产氧光合作用的证据有:}
        \begin{enumerate}
            \item \textbf{叠层石}:蓝细菌的化石;
            \item \textbf{矿物风化}:Fe$^{2+}$溶于水,氧化后成为Fe$^{3+}$不溶于水,因此土壤中铁的含量可以反映大气中氧的浓度;
            \item \textbf{碳掩埋}:光合作用在释放一分子氧气的同时会将一分子的碳固定到有机物中,因此地层中煤炭等物质含量随年代的变化可以反映大气氧气浓度的变化;
            \item 2-甲基藿烷:蓝细菌合成一种特有的化合物2-甲基菌何帕醇,在沉积物中转化为2-甲基藿烷。
        \end{enumerate}
        \item 光合作用出现的年代是通过$^{13}$C/$^{12}$C的比值判定的,因为光合作用更偏向于摄取$^{12}$CO$_2$,故光合作用出现后矿质中的$^{13}$C含量会升高。
        \item 影响大气中CO$_2$和O$_2$水平的因素包括\uline{岩石的化学风化}和\uline{生命体}。岩石的化学风化会将CO$_2$固定在土壤中。
    \end{itemize}
    \item 植物为人类提供赖以生存的氧气。
    \item 促进自然界的物质循环(C、O、N).
    \begin{itemize}
        \item \textbf{固氮细菌}和\textbf{固氮蓝藻}把大气中的游离氮,固定成植物能吸收的氨态氮,或经\textbf{硝化细菌}转化成硝态氮,供植物吸收。
        \item 动植物死亡后,尸体被细菌、真菌分解,又把氮以\uline{氨或铵形式}释放出来,后者可为植物利用。
    \end{itemize}
    \item 参与土壤形成,并为一切生物准备栖息的场所。
    \begin{itemize}
        \item 地球表面的土壤的形成,主要是由植物参与的。
        \item 植物吸收母质中有效矿物质,使养分成为有机态,固定在植物体中。
        \item 生物死亡后,尸体经分解形成腐殖质,改善土壤理化性质。
    \end{itemize}
    \item 为地球上一切生命提供食物、能源。植物正日益成为绿色环保的生物能源。
    \begin{itemize}
        \item \textbf{能源植物}:具有合成较高还原性烃的能力、可产生接近石油成分和可替代石油使用的产品的植物,以及 富含油脂的植物。
        \item 能源植物的种类:富集类似石油成分的能源植物、富集碳水化合物的能源植物、富集油脂的能源植物(大戟科、豆科、其他木本植物)。
        \item 能源植物研究热点:燃料乙醇、生物柴油。
    \end{itemize}
    \item 植物是天然的药物和保健食品。
    \begin{itemize}
        \item 常见药用植物:金银花、甘草、木麻黄、射干、黄花蒿(青蒿素)、红豆杉(紫杉醇)、三尖杉。
        \item 药用植物所含的化学成分:生物碱、苷类、挥发油、单宁等。
    \end{itemize}
    \item 植物可用作生物反应器。1986年,第一个人类重组蛋白——\textbf{生长激素}在烟草和向日葵中成功表达。
\end{enumerate}

\section{作物驯化}
\subsection{稻作概况}
\begin{enumerate}
    \item 普通野生稻驯化为栽培稻的过程中,由\uline{匍匐生长}变为\uline{直立生长}。
    \item 普通栽培稻分类:亚洲栽培稻、非洲栽培稻。
    \item 亚洲栽培稻的亚种分类:日本型和印度型,或粳和籼。
    \item 粳与籼的区别:
    \begin{enumerate}
        \item \textbf{粳米}\uline{支链淀粉}较多,因而\uline{口感较粘}(如糯米);
        \item \textbf{籼米}\uline{直链淀粉}含量比粳米高,因而\uline{口感较硬}。
    \end{enumerate}
    \item \textbf{渗入系}:通过系统回交和自交并利用分子标记辅助的手段使供体染色体片段渗入到受体亲本中。
\end{enumerate}

\subsection{水稻的育种突破}
\begin{enumerate}
    \item 第一次育种突破(\textbf{第一次绿色革命}):发生在20世纪50年代初,其主要特征是\uline{把高秆变矮秆}。\textbf{Norman Borlang}于1961年选育出第一批半矮秆、抗锈病小麦种并于次年在印度试种成功。矮秆水稻之父、半矮化水稻之父\textbf{黄耀祥}于1959年培育成世界上第一个耐肥、抗倒、高产的籼稻品种“广场矮”。
    \item 第二次育种突破:\textbf{野败不育基因CMS(cytoplasmic male sterility)}的发现与杂交水稻的成功。主要贡献者为杂交水稻之父——\textbf{袁隆平}。
    \item 第三次育种突破:分子标记辅助选择育种、转基因育种。
\end{enumerate}

\subsection{栽培植物及其野生祖先的主要差别}
\begin{enumerate}
    \item 栽培植物的巨大性;
    \item 栽培植物器官的异速生长;
    \item 自然传播手段的退化或丧失;
    \item 丧失保护机能;
    \item 化学成分的改变;
    \item 种子休眠性的减弱或丧失,成熟具有一致性;
    \item 生活周期的改变;
    \item 栽培植物类型的多样性。
\end{enumerate}