\chapter{植物的生长发育}
\section{植物细胞生长分化}
\begin{enumerate}
    \item \textbf{植物生长发育}:从受精卵发育开始,形成植物个体的整个过程。
    \item \textbf{植物生长发育周期}:从受精卵发育开始,形成植物个体,最后形成雌/雄配子体,可以称作一个发育周期。
    \item \textbf{生长}:指植物器官在体积、重量、数目等形态指标方面的增加,是一种量的变化。细胞水平上是通过原生质的增加,细胞分裂、伸长和扩大来实现的。整株水平上是根,茎,叶,花,果实和种子的体积增大和干重的增加。
    \item \textbf{发育}:是植物生长和分化的动态过程与总和。
    \begin{enumerate}
        \item 叶的发育:叶原基分化到长成成熟的叶片。
        \item 根的发育:根原基的发生到形成完整根系。
        \item 花的发育:茎端的分生组织形成花原基,转变成花蕾,到形成花序最后花蕾长大。
    \end{enumerate}
    \item 植物细胞分裂:末期由高尔基体分泌的小泡形成的细胞板实现细胞质分裂。
    \item 植物细胞分化过程四步模式:信号产生和感受$\to$分化调节基因表达$\to$结构和功能基因表达$\to$结构和功能分化成熟。
    \item 植物细胞分裂分化的特点:全能性。
\end{enumerate}

\section{植物发育——从胚胎到整体}
\subsection{高等植物生长发育的特点}
\begin{enumerate}
    \item \textbf{分生组织}:植物体内的一些体积较小、等直径的胚性细胞。
    \item \textbf{组织原细胞}:分生组织内衍生各种组织的原始细胞。
    \item 分生组织“自我存留”的性质:分裂的两个细胞,一个保留组织原细胞的特性,另一个进入特定分化进程。
    \item 分生组织的分类
    \begin{enumerate}
        \item \textbf{初生分生组织}:在胚胎发育过程中形成的分生组织,包括根和茎的顶端分生组织,以及侧生分生组织:原表皮层(分化形成表皮组织)、基本分生组织(分化形成皮层和内皮)和原形成层(分化形成维管组织和中柱鞘)。在多年生植物中仅存根和茎的顶端分生组织。
        \item \textbf{次生分生组织}:后期发育过程中形成的分生组织,包括叶腋分生组织、侧生分生组织中的维管形成层和表皮形成层,以及禾本科植物节间基部居间分生组织。
    \end{enumerate}
    \item 次生分生组织与初生分生组织在构造和性质上完全相同。
    \item 植物生长的有限性和无限性:
    \begin{enumerate}
        \item \textbf{有限性}:植物的组织器官到一定的阶段和大小时就停止生长发育,然后衰老、死亡。植物的叶片,花,果实等器官;
        \item \textbf{无限性}:营养器官的生长具有潜在的无限性,根,茎等器官。
    \end{enumerate}
    \item 植物的一年生和多年生习性:
    \begin{enumerate}
        \item 单次结实性:只开一次花,然后衰老死亡;
        \item 多次结实性:多年生植物在每次开花结实时,仍然保留大量的营养枝。
    \end{enumerate}
\end{enumerate}
\subsection{植物生长发育的控制}
\begin{enumerate}
    \item 基因水平的控制:胞内细胞信号过程,转录水平,转录后水平,翻译和翻译后水平;植物生长发育是基因程序性表达的结果。
    \item 激素水平的控制:胞间细胞通讯、胞内信号转导、基因表达调节。
    \item 环境的控制:胞外信号:光、温、水、重力、磁场、风、土壤的酸碱度,等。细胞接受环境信号,通过细胞内信号转导,调节基因表达,控制生长发育进程。
    \item 环境和激素最终通过调节基因表达来控制生长发育。
\end{enumerate}
\subsection{植物胚胎发育}
\begin{enumerate}
    \item \textbf{胚胎发生}:植物的生长发育是从\uline{子房胚囊}内的单细胞受精卵发育为多细胞胚胎这个过程开始的。
    \item \textbf{植物发育基本格式}:植物整体的轴向构造格式和器官径向构造格式。
    \begin{enumerate}
        \item \textbf{轴向构造格式}(整体):植物体呈现根茎两极模式,这种形态特点是植物极性构造的表现。
        \item \textbf{径向构造格式}(器官):根或茎的横截面各种组织呈现典型的同心圆排列形态:表皮,皮层,内皮层,中柱鞘和中柱。
    \end{enumerate}
    \item \textcolor{red}{胚胎发育的三个阶段}:
    \begin{enumerate}
        \item \textbf{球形胚}:\uline{轴向两极已经确定},并\uline{表现出径向构造格式}。
        \item \textbf{心形胚}:\uline{轴向构造已经形成}。
        \item \textbf{鱼雷形胚}:径向构造、轴向构造\uline{全部形成}。
    \end{enumerate}
\end{enumerate}
\subsection{种子的萌发}
\begin{enumerate}
    \item 种子是由\uline{受精胚珠}发育而来的,是脱离母体的延存器官。
    \item 严格地说,生命周期是从受精卵分裂形成胚开始的,但人们习惯上还是以种子萌发作为个体发育的起点,因为农业生产是从播种开始的。
    \item 双子叶植物:胚乳退化,子叶储藏营养。
    \item 单子叶植物:有胚乳。
    \item 通常以\uline{胚根突破种皮}作为萌发的标志。根据萌发过程中种子吸水量,即种子鲜重增加量的“快-慢-快”的特点,可把种子萌发分为三个阶段。
    \item 种子萌发过程中,酶系统形成,ABA减少,GA和IAA增加。
    \item 种子萌发的类型:
    \begin{enumerate}
        \item 子叶出土型:下胚轴伸长将子叶推出土,如蚕豆;
        \item 子叶留土型:上胚轴伸长,如菜豆、单子叶植物;
    \end{enumerate}
\end{enumerate}
\subsection{根系的生长分化}
\begin{enumerate}
    \item 根的结构特点:结构类型相同的组织细胞呈条状排列。从外到内:\uline{表皮-皮层-内皮层-中柱鞘-中柱}。
    \item 根中的所有组织都是从根尖分生组织中分裂产生的组织原细胞衍生而来。
    \item \textbf{静止中心}是根的干细胞群,可以自我更新,其功能在于维持或更新其周围的组织原细胞。
    \item 各种组织的原细胞分裂产生相应的组织:
    \begin{enumerate}
        \item 根冠柱原$\to$根冠柱;
        \item 根冠表皮原$\to$侧根冠和表皮;
        \item 皮层内皮层原$\to$皮层和内皮层;
        \item 中柱原$\to$中柱鞘和中柱。
    \end{enumerate}
    \item 根尖分生区的分裂:
    \begin{enumerate}
        \item 分化分裂:一般为平周分裂,形成条状组织的基础;
        \item 增殖分裂:一般为垂周分裂,增加条状组织的细胞数目。
    \end{enumerate}
    \item 侧根分化:中柱鞘或内皮层(因植物种而异)产生侧根原基,进而分化形成侧根。
\end{enumerate}
\subsection{茎的生长分化}
\begin{enumerate}
    \item 茎尖细胞的分化:茎尖产生许多侧生结构,包括叶原基、芽原基、花器官原基等。
    \item 原套原体理论:
    \begin{itemize}
        \item 原套和原体称为\textbf{组织形成层}。
        \item 表面一至数层排列整齐、较小的细胞为原套,进行垂周分裂,使茎尖的表面积增大。
        \item 原体细胞较大,可进行各个方向的分裂,使茎尖的体积增大。
    \end{itemize}
    \item 中央区周缘区理论:
    \begin{itemize}
        \item 中央区:中央母细胞区,细胞大,分裂慢。它产生茎中所有组织的原细胞。
        \item 中央区下——肋状分生区:分化为茎的内部组织(皮层、内皮和中柱)。
        \item 周缘区:细胞体积小,分裂快。形成叶、腋芽和茎的外层。中央区周缘区包含了原套和原体。
    \end{itemize}
    \item 茎分生组织分为:
    \begin{enumerate}
        \item 营养分生组织:分化叶片,腋芽,和维持茎尖无限生长;
        \item 成花分生组织:分化花器官原基(萼片,花瓣,雌蕊,雄蕊)。
    \end{enumerate}
\end{enumerate}
\subsection{叶的生长分化}
\begin{enumerate}
    \item 叶原基的形成及其分化:\textbf{叶原基}是顶端分生组织原套L1和L2层局部细胞(周缘分生组织区)分裂产生的。子细胞经平周分裂,突出表面形成叶原基,同时叶原基表层细胞进行垂周分裂增加表面积。 
    \item 叶原基的发生并不是随机的,在时间和空间上具有相当的确定性和精确性,并且叶原基发生形态具有种属特异性。叶原基或叶片在茎干上特定的排列状态称为\textbf{叶序}。叶片形态和叶序是植物分类的重要依据之一。
    \item 叶原基分裂分化产生叶片,叶片划分为
    \begin{enumerate}
        \item 近轴区:叶片上表面、朝向顶端分生组织中心的一面;
        \item 远轴区:叶片下表面、背向顶端分生组织中心的一面。        
    \end{enumerate}
\end{enumerate}
\subsection{植物生长发育的相关性}
\begin{enumerate}
    \item 营养物质:叶片光合同化物的下运、根部水分和矿物质的上运。
    \item 根冠信息交流——地上部分与地下部分的相关。ABA被认为是一种逆境信号,在水分亏缺时,根系快速合成并通过木质部蒸腾流将ABA运输到地上部分,调节地上部分的生理活动。如缩小气孔开度,抑制叶的分化与扩展,以减少蒸腾来增强对干旱的适应性。
    \item 顶端优势——主茎与侧枝的相关。植物的顶芽抑制侧芽生长的现象,称为“\textbf{顶端优势}” 。 
\end{enumerate}

\section{植物生殖器官发育}
\subsection{花芽分化和性别表达}
\begin{enumerate}
    \item 一般将花原基形成、花芽各部分的分化与成熟的过程,称作花器官的形成或\textbf{花芽分化}。花芽分化是从营养生长到生殖生长的过渡。植物一生分幼年期、成年期、生殖期。
    \item 花芽分化时的形态变化:\uline{生长点肥大高起},略呈半球形,与叶芽有明显区别。如果生长环境不合适,叶芽会停留在营养生长状态,不分化为花芽。
    \item 花芽分化时的生理生化变化:细胞代谢水平增高,有机物转化剧烈。可溶性糖、氨基酸、蛋白质增加,核酸的合成速度提高。
    \item 花器官形成所需要的条件:
    \begin{enumerate}
        \item 营养:碳水化合物、氮素、精氨酸和精胺、含磷化合物;
        \item 内源激素:CTK、乙烯、ABA——促进,GA——抑制,IAA——低浓度促进,高浓度抑制;
        \item 环境因子:光照(光合作用、光周期诱导)、温度、水分、矿质营养。
    \end{enumerate}
    \item 性别分化的调控因素:
    \begin{enumerate}
        \item 遗传因素:性别决定基因及与性别决定基因相互作用的\textbf{基本性基因}。
        \item 年龄:雌雄同株异花的植物,雄花往往出现在发育的早期,然后才出现雌花。 
        \item 环境条件:
        \begin{itemize}
            \item 光周期,调节开花,而且能控制性别表达和育性。植物继续处于诱导的适宜光周期下,促进多开\uline{雌花}。
            \item 温周期,较低的夜温与昼夜温差大时对许多植物的\uline{雌花}发育有利。
            \item 营养条件,水分充足、氮肥较多促进\uline{雌花}分化。
        \end{itemize}
    \end{enumerate}
    \item \textbf{同源异形突变}:花的某一重要器官位置发生了另一类器官替代的突变,如花瓣部位被雄蕊替代等。控制同源异型化的基因称为\textbf{同源异型基因}。
    \item \textcolor{red}{花形态建成遗传控制的“ABC模型”假说:}
    \begin{itemize}
        \item 典型的花器官具有四轮基本结构,从外到内依次为萼片、花瓣、雄蕊和心皮。
        \item 这四轮结构分别由A、AB、BC和C组基因决定。
        \item A组基因控制第1、2轮花器官的发育,其功能丧失会使第1轮花萼变成心皮,第2轮花瓣变成雄蕊。
        \item B组基因控制第2、3轮花器官的发育,其功能丧失会使第2轮花瓣变成萼片,第3轮雄蕊变成心皮。
        \item C组基因控制第3、4轮花器官的发育,其功能丧失会使第3轮雄蕊变成花瓣,第4轮心皮变成萼片。
        \item A、B双缺失:心皮完整,萼片残留。
        \item A、C双缺失:萼片、雄蕊、心皮残留。
        \item B、C双缺失:只剩萼片。
        \item A、B、C三缺失:花器官部分残留。
        \item \emph{SEP}s基因也很重要,与A/B/C基因互作,决定花器官形成
    \end{itemize}
\end{enumerate}

\subsection{雌雄配子体发育与授粉受精}
\begin{enumerate}
    \item 雄配子体——花粉粒的发育:孢原细胞$\to$小孢子母细胞$\overset{\text{减数分裂}}\longrightarrow$小孢子$\to$花粉粒。
    \item 花粉粒的结构:
    \begin{enumerate}
        \item 外壁:纤维素,角质,孢粉素,蛋白质(糖蛋白、酶、凝集素)
        \item 内壁:果胶质,胼胝质,蛋白质(水解酶)
        \item 原生质:营养核+2个精细胞 (雄配子)或1个生殖细胞(花粉管伸长时分裂为2个精细胞),液泡。
    \end{enumerate}
    \item 雌配子体——胚囊的发育:孢原细胞$\to$大孢子母细胞$\overset{\text{减数分裂}}\longrightarrow$四个大孢子$\to$功能大孢子确立。
    \item 功能大孢子三次分裂形成3个反足细胞、2个助细胞、1个卵细胞、1个极核(2倍型)。
    \item \textbf{雌蕊}:由一个或多个包着胚珠的\uline{心皮}连合成为雌性生殖器官。扁平的心皮闭合成雌蕊后,其上端为\uline{柱头},中间为\uline{花柱},下端为\uline{子房}。在子房中形成胚珠,胚珠内包含胚囊。
    \item 胚珠的构造:
    \begin{enumerate}
        \item 珠心:产生孢原细胞,形成胚囊;
        \item 珠被;
        \item 珠孔:花粉管进入胚囊的通道,但不是必须通道。有的植物花粉管从其他部位直接穿过珠被进入胚囊;
        \item 珠柄;
        \item 合点:珠心与珠被连和的部位,从子房胎座通过珠柄的维管束,经合点进入胚珠,为其提供养料。
    \end{enumerate}
    \item \textbf{授粉}:发育成熟的花粉落在雌蕊柱头上的过程。
    \item 花粉与柱头通过\uline{花粉外壁蛋白}和\uline{柱头乳突细胞表面蛋白}识别。
    \item 花粉萌发:花粉粒在柱头上经过识别, 从柱头的分泌物中吸收水分而水合,其内部压力增大,花粉粒内壁从外壁上的萌发孔向外突出形成花粉管。
    \item 花粉管的生长局限于顶端区域。
    \item \textbf{双受精}:一个精子与卵细胞融合发育成胚,另一个精子与极核融合发育成三倍体的胚乳。
    \item 受精后,二倍体胚和三倍体胚乳开始发育。花粉细胞的营养核、胚囊的2个助细胞、3个反足细胞全部消失。
    \item 受精后的生理生化变化:呼吸强度明显提高、生长素含量增加、大量物质运输。
\end{enumerate}
\subsection{种子发育和成熟}
\begin{enumerate}
    \item \textcolor{red}{胚的发育过程:}原胚期$\to$球形胚期$\to$心形胚期$\to$鱼雷胚期$\to$子叶期$\to$胚。
    \item 种子的发育时期:
    \begin{enumerate}
        \item 胚胎发生期:受精到胚形态初步建成;
        \item 种子形成期:胚、胚乳或子叶迅速生长;
        \item 种子成熟休止期:胚进入休眠期。        
    \end{enumerate}
    \item 种子发育过程中的生理生化变化:
    \begin{enumerate}
        \item 种子代谢和贮藏物质积累,分为淀粉种子、脂肪种子、蛋白质种子;
        \item 含水量\uline{降低},原生质由溶胶状态转变为凝胶状态。
        \item 内源激素的动态变化:CTK、GA/IAA、ABA依次出现高峰。
    \end{enumerate}
\end{enumerate}