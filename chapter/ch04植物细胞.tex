\chapter{植物细胞}
\section{细胞概述}
\begin{enumerate}
    \item 原核细胞与真核细胞的比较:
    \begin{table}[h]
        \centering
        \begin{tabular}{C{3.4cm}C{3.7cm}C{3.7cm}}
            \toprule
            &\textbf{原核细胞}&\textbf{真核细胞}\\
            \midrule
            \textbf{细胞膜}&有(多功能性)&有\\
            \textbf{膜包被的细胞器}&无&有\\
            \textbf{核糖体}&70S(50S+30S)&80S(60S+40S)\\
            \textbf{光合作用结构}&蓝细菌含有叶绿素a的膜层结构,细菌有菌色素&植物有叶绿素a和叶绿素b\\
            \textbf{核外DNA}&质粒&线粒体DNA、叶绿体DNA\\
            \textbf{细胞壁}&氨基糖和壁酸&纤维素和果胶\\
            \textbf{细胞骨架和蛋白酶体}&有&有\\
            \bottomrule
        \end{tabular}
    \end{table}
    \item 光合作用细菌:蓝细菌、原绿菌、紫色细菌、绿色细菌。一般认为,光合细菌是绿色植物的祖先:光合细菌共生于真核细胞之中,经过长期演化,形成叶绿体,产生了有光合作用能力的植物细胞。
    \item 植物细胞壁的组成:\uline{初生壁}、\uline{次生壁}、\uline{胞间层}、\uline{胞间连丝}。
    \item 双层膜细胞器:线粒体、叶绿体、细胞核、\uline{质体}。
    \item 单层膜细胞器:高尔基体与高尔基体小泡、内质网、液泡、溶酶体、\uline{微体}。
    \item 无膜细胞器:核糖体、\uline{微丝}、\uline{微管}、\uline{中间纤维}、\uline{蛋白酶体}。
    \item \textcolor{red}{植物细胞与动物细胞的区别}:
    \begin{enumerate}
        \item 植物细胞特有:细胞壁、质体、乙醛酸体、液泡、叶绿体、胞间连丝;
        \item 动物细胞特有:中心体。
        \item 分裂方式:植物细胞质分裂是由高尔基体分泌小泡形成细胞板完成的,动物细胞质分裂是由胞质分裂环收缩完成的。
        \item 植物细胞的分裂能力和全能性强于动物细胞。
    \end{enumerate}
\end{enumerate}

\section{细胞的膜系统}
\begin{enumerate}
    \item 质膜的化学成分:
    \begin{enumerate}
        \item 膜脂:磷脂、糖脂、胆固醇;
        \item 膜蛋白:外在膜蛋白、内在膜蛋白。
    \end{enumerate}
    \item 不同细胞器的膜的脂类和蛋白质组成有差异。
    \item \textbf{内膜系统}:在细胞中存在许多由膜包围而形成的细胞器,这些细胞器组成一个在结构、功能上具连续性的膜系统。
    \item 微体:
    \begin{enumerate}
        \item 过氧化物酶体:内含氧化酶和过氧化氢酶;
        \item 乙醛酸循环体:\uline{将脂肪转化为糖}。
    \end{enumerate}
    \item 微体与溶酶体、液泡\uline{共同起源于内质网},形态类似(同为小泡),难于分辨。(区分:溶酶体起源于高尔基体。)
    \item 液泡来源于内质网,通过吞噬细胞质和互相融合形成成熟液泡。液泡的功能:
    \begin{enumerate}
        \item 增加植物细胞膜系统的面积;
        \item 提高细胞的膨压;
        \item 细胞内重要的物质储藏库;
        \item \uline{参与细胞物质代谢和信号转导}。
    \end{enumerate}
\end{enumerate}

\section{叶绿体和线粒体}
\begin{enumerate}
    \item 叶绿体/线粒体的基因组只能编码5-10\%组成它们的蛋白质,所以,核基因组在很大程度上决定这两个细胞器的建成和功能。但是,这两个细胞器可以向细胞核\uline{发送“反向信号”},\uline{调节有关的核基因表达}。
    \item 叶绿体由双层膜包被,基质中悬浮着由单位膜封闭形成的扁平小囊,称为\textbf{类囊体}。圆盘型类囊体堆叠为\textbf{基粒},组成基粒的类囊体称为\textbf{基粒类囊体},其片层称为\textbf{基粒片层}。贯穿于基粒之间的没有发生堆叠的类囊体称为\textbf{基质类囊体},其片层称为\textbf{基质片层}。光合作用的光反应在类囊体上进行。
\end{enumerate}

\section{26S蛋白酶体}
\begin{enumerate}
    \item \textbf{泛素}是含有76个氨基酸的小分子蛋白,用来标记需要降解的蛋白质,形成泛素-蛋白复合体,继而被26S蛋白酶体识别并分解掉。
    \item 蛋白质降解的\textbf{泛素-蛋白酶体途径}:泛素对靶蛋白的标记需要通过\uline{三种泛素酶系(E1,E2,E3)}的依次作用,最终将多聚的泛素连接到底物上,然后\textbf{26S蛋白酶体}识别被泛素化的蛋白使之降解。
    \item 泛素C末端的\uline{甘氨酸G76}为E1、E2或E3的结合位点,其羧基\uline{与E1、E2或E3的半胱氨酸形成高能硫酯键};另外,\uline{与底物蛋白或另一泛素的赖氨酸形成异肽键}。\uline{赖氨酸(K)}的侧链氨基与另一泛素的C末端的甘氨酸形成异肽键。
    \item 泛素级联反应中的重要组分:
    \begin{enumerate}
        \item \textbf{E1:泛素活化酶},是泛素与底物结合所需要的第一个酶,但是对靶蛋白的特异性几乎没有影响,它的作用为水解ATP, 通过本身的半胱氨酸残基与泛素的羧基末端形成高能硫酯键而\uline{激活泛素}。
        \item \textbf{E2:泛素结合酶},泛素分子经过转硫醇反应从E1半胱氨酸残基转移到E2活化的半胱氨酸位点,然后将泛素转移到E3上,或者在E3的帮助下将泛素分子直接转移到靶蛋白上。
        \item \textbf{E3:泛素-蛋白连接酶},负责选择和识别需要泛素化的蛋白质。
        \begin{itemize}
            \item E3可以接受E2转来的泛素(同样以硫酯键而结合,形成所谓“自泛素化”),进而催化泛素与靶蛋白的赖氨酸ԑ-氨基形成异肽键,从而使底物泛素化;
            \item 在很多情况下,E3结合E2后,不进行“自泛素化”,而是诱导E2上结合的泛素直接转移到底物蛋白上去;
            \item 激活的泛素可以在E3的作用下,与已和底物蛋白结合的泛素的赖氨酸残基形成异肽键,产生多聚泛素(一个泛素的C-端与另一个泛素的赖氨酸相连接)。
        \end{itemize}
    \end{enumerate}
    \item \textbf{26S蛋白酶体}在植物细胞的胞质和核中都存在(动物细胞中也主要是细胞质和核定位),分子量为2000KDa,由圆柱形的20S\textbf{核心催化颗粒(core particle, CP)}和两个V形的 19S的\textbf{调控粒子(regulatory particle, RP)}组成;CP具有\uline{水解肽键}的功能,RP的主要功能是\uline{辅助识别被泛素化的底物},\uline{使之解折叠},\uline{去除ubiqutin链},\uline{并指导解折叠的底物进入CP从而使其降解}。
\end{enumerate}

\section{细胞骨架}
\begin{enumerate}
    \item \textbf{微丝}是由单体肌动蛋白聚合成的直径7-8nm的螺旋丝状结构,又称\textbf{F-actin},相应地单体肌动蛋白又称\textbf{G-actin}。
    \item 微丝的功能:
    \begin{enumerate}
        \item 参与胞质运动;
        \item 参与物质运输、细胞感应和细胞生长调节;
        \item \uline{参与顶端生长,可能主要是促进物质向顶端的运输用于细胞膜和细胞壁合成与扩张}。
    \end{enumerate}
    \item \textbf{微管}是由α-和β-微管蛋白组成异二聚体,异二聚体聚合形成的直径约25nm的空心管状结构。直列上连在一起的二聚体称为\textbf{原丝体},大多数微管由13个原丝体构成。13条原丝体定向平行排列形成微管(可发生可逆的解聚或聚合),通过微管上的许多结合位点可与质膜、核膜、内质网发生联系。
    \item 微管的功能:
    \begin{enumerate}
        \item 控制细胞分裂,\uline{参与细胞板的形成};
        \item 通过控制细胞壁的形成保持细胞形状:\uline{周质微管决定纤维素微纤丝在细胞外沉积的走向}。在植物不同的细胞中,可观察到紧贴质膜之内的微管与紧贴质膜之外的纤维素微纤丝的方向一致。        ;
        \item 参与细胞运动(如:纤毛运动,鞭毛运动,纺锤体和染色体运动)、细胞内物质运输、信息传递。
    \end{enumerate}
    \item \textbf{中间纤维}由丝状亚基组成。亚基双股超螺旋的二聚体形成四聚体,进一步形成圆柱状的中间纤维。
    \item 中间纤维的功能:
    \begin{enumerate}
        \item 支架作用:中间纤维可从核骨架向细胞膜延伸,从而提供了一个起支架作用的细胞质纤维网,使细胞保持空间的完整性,并与细胞核的定位有关。
        \item 参与细胞发育与分化:中间纤维与细胞发育、分化、mRNA等的运输有关。
    \end{enumerate}
\end{enumerate}

\section{细胞壁}
\subsection{细胞壁的化学组成}
\begin{enumerate}
    \item \textcolor{red}{植物细胞的主要化学成分:}\uline{纤维素}、\uline{半纤维素}、\uline{果胶}。
    \item \textbf{纤维素}:约占初生壁的20\%-30\%,是构成细胞壁的主要成分。纤维素由1000-10000个D-葡萄糖残基以β-1,4-糖苷键结合形成纤维素分子链,不分支、不溶于水。
    \item \textbf{半纤维素}:约占20\%-25\%。由不同种类的糖聚合而成的异质多聚糖(中性和酸性)。半纤维素的结构较复杂,在化学结构上与纤维素无关。 不同来源的半纤维素,组成成分也不相同:有的只有一种单糖缩合而成,如聚甘露糖或聚半乳糖;有的有几种单糖缩合而成,如木聚糖、阿拉伯糖、半乳聚糖等。
    \item \textbf{果胶}:约占10\%-35\%。异质多聚糖(α-1,4–D-半乳糖醛酸),是由半乳糖醛酸组成的多聚体。胞间层基本上由果胶物质组成,果胶使相邻的细胞粘合在一起。根据其结合情况及理化性质,又可分为三类:果胶酸、果胶和原果胶。 
    \item 蛋白质:5\%-10\%
    \begin{enumerate}
        \item 参与细胞壁结构的蛋白质:如\textbf{伸展蛋白(extensin)};
        \item 与细胞壁组建或调节壁特性有关的酶类:如纤维素酶、过氧化物酶类;
        \item 识别或调节蛋白:如糖蛋白(参与细胞防卫反应的\textbf{凝集素})。
    \end{enumerate}
\end{enumerate}
\subsection{细胞壁形成}
\begin{enumerate}
    \item 细胞核、高尔基体和内质网协同作用,合成\textbf{造壁物质}(中胶层,主要为果胶质)。
    \begin{itemize}
        \item 细胞有丝分裂晚后期,母细胞的赤道板上有不规则的分泌小囊泡(这些小囊泡是由高尔基体和内质网分泌形成的),借助于\uline{微管},排列成一排,形成\textbf{成膜体(phragmoplast)}。
        \item 小泡经过融合,小泡内的造壁物质成分释放到质膜外形成细胞壁,融合的小泡的膜则形成质膜。
    \end{itemize}
    \item 纤维素的合成:在质膜上存在一种\uline{蛋白复合体/终端复合体},包含多个纤维素合成酶单元,催化纤维素合成。
    \item 微纤丝的组装:
    \begin{itemize}
        \item 每一个纤维素合成酶合成一条纤维素分子,成对的纤维素分子之间由葡萄糖残基上的-OH基所形成的氢键相连结形成\textbf{微团(micell)}结构,每个微团再以氢键与周围的微团结合,形成\textbf{微纤丝(microfibril)},微纤丝又组成\textbf{大纤丝(macrofibril)}。因为纤维素的这种结构非常牢固,使细胞壁具有\uline{高强度}和\uline{抗化学降解}的能力。
        \item 质膜下的\uline{微管}指导\uline{微纤丝的排列方向}。
        \item 微纤丝周围充满着衬质,衬质包括半纤维素和果胶质。胞间层主要由果胶组成。初生壁中纤维素微纤丝以一定的方式排列,半纤维素和果胶以氢键形式将微纤丝连接起来,使初生壁形成一种非常牢固的结构。
    \end{itemize}
\end{enumerate}
\subsection{细胞壁的结构}
\begin{enumerate}
    \item 成熟的细胞壁由\uline{胞间层}、\uline{初生壁}和\uline{次生壁}组成。
    \item \textbf{初生壁}:较薄,有弹性可以随细胞的伸长而延伸。如分生组织细胞、胚乳细胞等。
    \item \textbf{次生壁}:次生壁内侧沉积\uline{角质}、\uline{木质素}、\uline{硅质}和\uline{结构蛋白},层与层之间经纬交错。一般较厚,而且坚硬,常出现在起机械支持和输导作用的植物细胞中,如某些特化的细胞:纤维细胞、厚壁细胞、管胞、导管。
\end{enumerate}
\subsection{细胞壁的功能}
\begin{enumerate}
    \item 维持细胞的形状。
    \item 在细胞的伸长生长中,细胞壁的弹性大小对细胞的\uline{生长速率}起重要的调节作用。
    \item 细胞壁有很高的机械强度,可提高细胞对外界机械伤害的抵抗力。
    \item 细胞壁是产生\uline{膨压}的必要因素,参与水分平衡。
    \item 细胞壁属于质外体空间,参与水分和矿物质运输。
    \item 参与一些特殊细胞的细胞运动,例如保卫细胞的开关。
    \begin{itemize}
        \item 肾形保卫细胞的内壁(靠气孔一侧)厚而外壁薄,微纤丝从气孔呈扇形辐射排列。
        \item 当保卫细胞吸水膨胀时,较薄的外壁易于伸长,向外扩展,但微纤丝难以伸长,于是将力量作用于内壁,把内壁拉过来,于是气孔张开。
    \end{itemize}
    \item 细胞壁是植物细胞的\uline{天然屏障},加强植物细胞抵御病害和逆境的能力。
    \item 细胞壁是植物细胞的外被,通过\uline{和细胞膜上的蛋白互作},将环境信号传递给细胞内。
\end{enumerate}

\section{胞间连丝和细胞间联络}
\subsection{胞间连丝的概念}
\begin{enumerate}
    \item 细胞的原生质膜突出,穿过细胞壁与另一个细胞的原生质膜连在一起。这种穿过细胞壁,沟通相邻细胞的原生质细丝称为\textbf{胞间连丝},是相邻细胞间穿通细胞壁的细胞质通路。
    \item 胞间连丝见于\uline{所有的高等植物}、某些低等植物如有些\uline{藻类},以及\uline{真菌}。
    \item 由胞间连丝将原生质连成一体的体系称为\textbf{共质体(symplast)}。胞间连丝将整体植物的原生质体连为一个整体,所以,植物周身的共质体是连续的。
    \item 细胞壁及细胞间隙等空间以及导管称为\textbf{质外体(apoplast)}。
    \item 动物细胞之间没有像植物细胞胞间连丝这样直接联系的通道。间隙连接是动物细胞中通过\textbf{蛋白质连接子}进行的细胞间连接。
\end{enumerate}
\subsection{胞间连丝的结构}
\begin{enumerate}
    \item 胞间连丝孔道外围:相邻\uline{细胞膜}延伸形成的管道。
    \item 管道中央:\uline{内质网}压缩成的连丝小管。\textbf{连丝小管}:内质网压缩成的狭窄小管,其两边分别与两边细胞的内质网膜相连,是封闭的。
    \item \textbf{孔环}:连丝小管与连丝外膜间的环状结构,是可能的物质扩散通道。
    \item \textbf{蛋白辐}:丝状蛋白和球形蛋白的复合体,将连丝小管与胞间连丝外围膜联系起来。
\end{enumerate}
\subsection{胞间连丝的功能}
\begin{enumerate}
    \item 物质运输:包括可溶性物质、生物大分子、甚至细胞核发生胞间运输,即\uline{“核穿壁”现象}。胞间连丝与动物细胞的间隙连接有许多相同之处。
    \item 胞间连丝可有\uline{开放态}和\uline{封闭态}的转变。开放态时,其孔道大小也可以变化。正常情况下,间隙连接允许1kD以下的分子渗透,也能让离子自由通过。与间隙连接不同的是,胞间连丝的孔\uline{能够扩张},允许大分子,包括蛋白质和RNA分子通过。
    \item 病毒RNA在运动蛋白的帮助下通过胞间连丝侵染邻近的细胞。有两种模型:
    \begin{enumerate}
        \item 运动蛋白结合在病毒RNA上,由运动蛋白将RNA带到相邻的细胞;
        \item 几种蛋白共同作用,穿过胞间连丝进入邻近细胞。
    \end{enumerate}
    \item 在病菌侵染时,植物可以关闭胞间连丝,这是一个防卫机制。水杨酸调节\textbf{Remorin蛋白}(一种定位于细胞质膜的植物专一性蛋白)依赖的膜脂重组,介导胞间连丝的关闭。
\end{enumerate}