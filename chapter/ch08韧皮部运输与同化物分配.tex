\chapter{韧皮部运输与同化物分配}
\section{韧皮部结构和运输特点}
\begin{enumerate}
    \item \uline{韧皮部}是同化物运输的主要途径。同化物在韧皮部中可向上、向下运输,运输的方向取决于库(生长中心)的位置。
    \item 韧皮部的结构:韧皮部在维管组织的向外一侧,主要由\uline{筛分子}、\uline{伴胞}和\uline{薄壁细胞}组成,在维管束的外周常包围着一层鞘,称\textbf{维管束鞘}。
    \item 单子叶植物和双子叶植物维管束差异:
    \begin{enumerate}
        \item 单子叶植物:维管束多独立分布于基本组织间。一年生禾本科植物只有初生结构,没有次生结构。
        \item 双子叶植物:连成一环。多年生树木初生结构消失,只有次生韧皮部和次生木质部。
    \end{enumerate}
    \item \textbf{筛分子}是筒状细胞,有线粒体和滑面内质网,有完整质膜。但是\uline{没有细胞核}。筛管分子组成筛管。筛管分子的壁上形成多孔的特化筛域——\textbf{筛板}。
    \item \textbf{伴胞}:在筛管周围,通常具有浓厚的细胞质、大量的线粒体,有细胞核。在筛管分子和伴胞之间有大量的胞间连丝。
    \item 伴胞的RNA和蛋白质可以运输到无细胞核的筛分子中,可能指导筛分子的功能。由于筛管分子和伴胞之间在结构和功能上的密切联系,通常把两者作为一个功能单位看待,称为\textbf{筛管分子-伴胞复合体(SE-CC complex)}。
    \item 韧皮部薄壁细胞:与其他组织中的薄壁细胞类似,细胞壁较薄,液泡很大,但是通常比普通薄壁细胞更长一些。可能有溶质和水的储存和运输的功能。
    \item 韧皮部运输的物质:水、干物质(非还原糖、含氮有机物、无机离子)。
    \item 韧皮部运输的方向:\uline{从源到库}。源:指生产同化物以及向其他器官提供营养的器官,如成熟叶片、种子萌发时的子叶或胚乳组织。库:消耗或积累同化物的接纳器官,如幼叶、根、花、果实、种子等。
    \item \textbf{源库单位}:同化物供求上有对应关系的源与库称为源-库单位。植物体内存在多源多库,同化物在源库器官中的运输存在空间和时间上的调节和分工。
    \item 源库间运输的规律:
    \begin{enumerate}
        \item 空间上就近运输;
        \item 优先在有维管束相连的源库间运输(维管束的并接:当维管束因植物受到伤害或修剪被切断时,韧皮部运输会发生改变);
        \item 向生长中心运输。
    \end{enumerate}
\end{enumerate}

\section{压力流动学说}
\begin{enumerate}
    \item \textcolor{red}{压力流动学说的内容:}
    \begin{itemize}
        \item 源端装载:在源端韧皮部进行溶质的装载$\to$溶质分子进入筛管分子$\to$细胞渗透压下降$\to$水势下降$\to$木质部的水顺着水势梯度进入筛管分子$\to$源端筛管分子膨压上升。
        \item 库端卸载:在库端韧皮部进行溶质的卸出$\to$溶质分子流出筛管分子$\to$细胞渗透压上升$\to$水势上升$\to$筛管的水顺着水势梯度流出,进入木质部$\to$库端筛管分子膨压下降。
    \end{itemize}
    \item 动力来源:源端(膨压高)到库端(膨压低)的膨压梯度推动韧皮部液流运动。
    \item 溶质在筛管中是随\uline{集流}而运动的,筛管内的集流是靠源端和库端渗透势引起的膨压差所建立的压力梯度来推动的。
    \item 压力流动学说的要点:
    \begin{enumerate}
        \item 同化物在筛管内运输是由源、库两侧筛管分子-伴胞复合体内渗透作用所形成的\uline{压力势梯度}所驱动的。
        \item 压力梯度的形成则是由于源端光合同化物不断向筛管分子-伴胞复合体进行装载,库端同化物不断从筛管分子-伴胞复合体卸出,以及韧皮部和木质部之间水分的不断再循环所致。
        \item 只要源端光合同化物的韧皮部装载和库端光合同化物的卸出过程不断进行,源、库间就能维持一定的压力梯度,在此梯度下,光合同化物可源源不断地由源端向库端运输。
        \item 在韧皮部的运输系统中,\uline{筛板极大的增加了筛管对水流的阻力},这种阻力对建立和维持源-库两端的膨压差是必须的。
    \end{enumerate}
\end{enumerate}

\section{韧皮部源端装载和库端卸出机制}
\subsection{韧皮部源端的装载}
\begin{enumerate}
    \item 韧皮部的装载:包括光合产物从成熟叶片中的\uline{叶肉细胞的叶绿体}运送到\uline{筛管分子-伴胞复合体}的整个过程。
    \item 韧皮部装载的步骤:
    \begin{enumerate}
        \item 光合作用中形成的\uline{磷酸丙糖}从叶绿体运到细胞质中,转化为\uline{蔗糖}。
        \item 蔗糖从叶肉细胞转移到叶片小叶脉筛管分子附近。这一途径的距离通常为2-3个细胞直径,称为\textbf{短距离运输途径}。
        \item 蔗糖转运到筛管分子中,称为\textbf{筛管分子装载}。
    \end{enumerate}
    \item 蔗糖及其他溶质进入筛管后,会随集流运出源器官,这个过程称之为\textbf{输出}。
    \item 从源经过维管系统到库的运输称为\textbf{长距离运输},由源库之间筛分子膨压差所推动。
    \item \textcolor{red}{共质体途径和质外体途径:}
    \begin{enumerate}
        \item 共质体途径:整个途径的细胞间都具有胞间连丝。
        \item 质外体途径:质外体途径并不意味着全过程都是在质外体中进行,只要在此途径中任何步骤必须经过质外体,那么这个途径就是质外体途径的韧皮部装载。
    \end{enumerate}
    \item 韧皮部装载类型的决定因子是细胞间是否存在共质体通道。筛管分子-伴胞复合体与周围细胞间胞间连丝的密度:
    \begin{enumerate}
        \item 有大量胞间连丝存在——共质体装载;
        \item 有中等密度的胞间连丝存在——共质体/质外体装载);
        \item 几乎无胞间连丝——质外体装载。
    \end{enumerate}
    \item 蔗糖可采取质外体途径或共质体途径,而棉子糖、苏水糖只采取质外体途径。
    \item 质外体途径的韧皮部装载机制:
    \begin{itemize}
        \item 质外体途径的蔗糖装载是需能的过程。
        \item 蔗糖/质子共转运蛋白:运输蔗糖与质子传递相偶联。 筛分子/伴胞质膜上的H$^+$-ATP酶产生电化学势梯度,推动载体将H$^+$的向内扩散与蔗糖的向共质体的转运偶联起来,称为\textbf{蔗糖/质子共转运}。属于\uline{次级主动运输}。
        \item 实际上,在筛分子-伴胞复合体装载前,必须从叶肉细胞中卸出。卸出的跨膜机制目前还是未知。
    \end{itemize}
    \item 共质体途径的韧皮部装载机制:共质体途径装载是通过胞间连丝进行的,是一种\uline{由浓度梯度推动}的装载。所以,细胞间存在胞间连丝是共质体运输的必要条件。
\end{enumerate}
\subsection{韧皮部库端的卸出}
\begin{enumerate}
    \item \textbf{韧皮部卸出}:韧皮部进行输出的同化物在库端被运出韧皮部并被邻近生长或储存组织所吸收的过程。
    \item 韧皮部输出过程也分为质外体和共质体路径。
    \item 植物器官发育过程中可能发生装载和卸出的转换。例如,幼小的叶片是接收同化物的库,在叶片发育的过程中逐步转化为源。
    \item 卸出途径的研究手段:
    \begin{enumerate}
        \item CFDA:不发荧光,非极性,可跨膜进入活细胞。进入活细胞后经非特异酯酶分解为CF,发荧光,极性分子,不能跨膜泄漏。
        \item 病毒运动蛋白+荧光蛋白:进入细胞后经胞间连丝在活细胞中运动。
    \end{enumerate}
    \item 葡萄果实发育前期果实中糖浓度很低,韧皮部采用共质体卸载路径可能有利于浓度梯度推动的卸出。而果实始熟后糖迅速积累,这时韧皮部质外体卸载有利于果实中逆浓度积累糖。
\end{enumerate}

\section{同化物的配置和分配}
\begin{enumerate}
    \item 植物将光合固定的碳调配到不同代谢途径称为\textbf{配置}。同化物配置与源叶输出和库器官同化物积累分配直接相关联。
    \item 光合叶片中同化物的配置:
    \begin{enumerate}
        \item 储存:光合固定的碳用于\uline{合成储存化合物}:大多数植物白天在叶绿体中合成和储存淀粉,夜间淀粉被动员用于输出。
        \item 利用:光合固定的碳可以\uline{被光合细胞自身所利用}。
        \item 运输:光合固定的碳\uline{合成用于运输的糖},然后被运输到各种库组织中。
    \end{enumerate}
    \item 库中的配置:储存、利用。
    \item 配置的调节:源叶中同化物配置的控制点是\uline{磷酸丙糖}配置。磷酸丙糖有三个去向:加入光合C3循环、进行淀粉合成、进行蔗糖合成。蔗糖合成过程中的关键酶是蔗糖磷酸合成酶和果糖-1,6-二磷酸酶。叶绿体淀粉合成中的关键酶是ADPG焦磷酸化酶(催化ADPG合成),这些酶是协调淀粉和蔗糖合成的控制点,也是源叶中同化物配置的关键调节点。这些代谢节点的调节(同化物配置调节)与源叶输出和库器官同化物积累分配直接相关联。
    \item 植物体中光合同化物有规律地向各库器官输送的模式称为\textbf{分配}。同化物的分配直接影响到植物生长和经济产量。
    \item 同化物分配的影响因素:源库间距离、维管束走向、横截面积及库器官的竞争能力。
    \item 库器官的类型:
    \begin{enumerate}
        \item \textbf{使用库}:是指大部分输入的同化物被用于生长的组织。在大部分使用库中,韧皮部与周围细胞有较多的胞间连丝,韧皮部的卸出通常采用共质体的途径。由于库细胞中同化物被不断代谢利用,可以保持韧皮部细胞与库细胞之间从高到低的同化物浓度差,有利于浓度梯度推动的共质体卸载。
        \item \textbf{储藏库}:是指大部分输入的同化物用于储藏的组织和器官,如果实、块茎、块根等。在许多储藏库中,韧皮部与周围细胞间有许多胞间连丝存在,韧皮部筛管分子卸出以共质体途径为主,随后的筛管分子后运输过程中经历了同化物卸出到质外空间,再通过质外体装载进入库细胞的过程。这实际上是一种质外体卸出路径。
    \end{enumerate}
    \item 库器官对同化物的竞争能力可能决定了植物体内同化物的分配。库的储藏和代谢输入糖的能力越强,从源争夺同化物的能力也就越强。这种竞争决定了运输糖在各种库组织间的分配。
    \item 库对同化物的竞争能力用\textbf{库强度}表示。
    \[
        \text{库强度}=\text{库容量}\times\text{库活力}
    \]
    \begin{itemize}
        \item \textbf{库容量}:库组织的总量;
        \item \textbf{库活力}:单位库组织吸收同化物的速率。
    \end{itemize}
    \item 库源的协调:
    \begin{enumerate}
        \item 源是库的供应者。源强会为库提供更多的光合产物。
        \item 库对源有反馈调节作用库强则能促进源中蔗糖的输出速率,并促进源叶光合速率。库弱抑制同化物输出,抑制叶片光合作用。库小源大:限制光合产物的输送分配,抑制光合作用。库大源小:促进光合作用,但当库的需求远超过源的负荷时,导致\uline{胁迫输送},引起库的空瘪和叶片早衰。
        \item 源库之间的信号传递:通过植物激素。糖水平影响光合作用和糖代谢有关基因的表达。糖不足,促进光合作用、储藏物动员输出有关基因的表达;糖富足,促进与储藏利用碳水化合物有关基因的表达。
    \end{enumerate}
    \item 同化物的\textbf{再分配}:植物体内已经同化的物质,除了已构成细胞壁这样的骨架物质外,其他物质——不论是有机的还是无机的——都可进行再度分配和再度利用。
    \item 同化物再分配的途径是\uline{韧皮部}。
\end{enumerate}