\chapter{光周期与开花}
\begin{enumerate}
    \item 植物的\textbf{光形态建成(photomorphogenesis)}:与暗下生长黄化而柔弱不同,植物在光下生长正常而健壮。与光合作用相比,这种现象依赖于\uline{较弱的}、\uline{短时的}光照,是一种\uline{低能量}反应。
    \item \textbf{光周期(photoperiod)}:自然界一昼夜间的光暗交替。经过漫长进化过程,生长在地球上不同地区(不同光周期下)的植物,其生长发育对光周期呈现出适应性变化。
    \item \textbf{光周期现象(photoperiodism)}:对光周期发生反应的现象。光周期现象是“植物光形态建成”的重要内容之一:植物的\uline{成花(最重要)}、休眠、落叶以及鳞茎、块茎、球茎等地下储藏器官的形成都受日照长度的影响。
\end{enumerate}

\section{植物成花对光周期的反应}
\begin{itemize}
    \item 1920年,美国马里兰州,农业部Beltsville农业试验站Garner和Allard发现,短日照是烟草开花的关键条件。
    \item 成花反应光周期类型:
    \begin{enumerate}
        \item \textbf{长日植物}:指在24小时昼夜周期中,日照长度必须长于一定时数,才能成花的植物,延长光照可促进和提早开花;相反,如延长黑暗则推迟开花或不能成花。
        \item \textbf{短日植物}:指在24小时昼夜周期中,日照长度短于一定时数才能成花的植物。对这些植物适当延长黑暗或缩短光照可促进和提早开花,如延长日照则推迟开花或不能成花。
        \item \textbf{日中性植物}:成花对日照长度/光周期不敏感,在任何长度的日照下均能开花。
        \item \textbf{长-短日植物}:开花要求先长日后短日的双重日照条件。
        \item \textbf{短-长日植物}:这类植物开花要求先短日后长日的双重日照条件。     
        \item \textbf{中日照植物}:只有在某一定中等长度的日照条件下才能开花,而在较长或较短日照下均保持营养生长状态的植物。
        \item \textbf{两极光周期植物}:与中日照植物相反,这类植物在中等日照条件下保持营养生长状态,而在较长或较短日照下才开花,如狗尾草等。
    \end{enumerate}
    \item \textbf{临界日长}:对光周期敏感的植物对日照长度的要求都有一定的临界值,或说是植物成花所需的极限日照长度,即临界日长。长日照植物开花需日照长度长于某一临界日长;短日照植物开花则要求短于某一临界日长。
    \item 植物开花对日长反应有的严格,有的不严格,分为\textbf{绝对长(短)日植物}和\textbf{相对长(短)日植物}。
    \item 同种植物的不同品种对日照的要求可以不同。\textcolor{red}{通常早熟品种多为长日或日中性植物,晚熟品种多为短日植物。}
    \item \textbf{临界暗期}或\textbf{临界夜长}:是指在光暗周期中,短日植物能开花的最小暗期长度或长日植物开花的最大暗期长度。
    \item \textbf{暗期决定论}:植物是通过检测光暗周期中的暗期长度来感知日长的变化。\uline{暗期间断实验},证明了暗期长度的决定作用。 
\end{itemize}

\section{植物光周期诱导的机理}
\begin{enumerate}
    \item \textbf{光周期诱导现象}:植物在达到一定的生理年龄时,经过\uline{足够天数的适宜光周期处理},以后即使处于不适宜的光周期下,仍然能保持这种刺激的效果而开花,称作光周期诱导。
    \item 短于诱导周期的最低天数,不能诱导植物开花,增加光周期诱导的天数可\uline{加速花原基的发育},花的数量也增多。
    \item 光周期诱导机理:光敏色素(Pfr/Pr)与成花素。\textbf{滴漏式测时}
    \[
        \text{Pr(非活性形式)}\overset{\text{红光}}{\underset{\text{远红光}}\rightleftharpoons}\text{Pfr(活性形式)}
    \]
    Pfr/Pr比值大,促进长日植物成花;Pfr/Pr比值小,促进短日植物成花。
    \item 两类光敏色素:
    \begin{enumerate}
        \item Phy I:以二聚体形式存在。吸收红光由Pr转变为Pfr后不稳定,迅速降解(DNA转录被抑制、mRNA被降解、蛋白质被泛素化降解),且在光下很少合成,\uline{在暗中合成并积累}。拟南芥PhyA为此类。
        \item Phy II:以二聚体形式存在。吸收红光由Pr转变为Pfr后稳定,在光下和暗中都能合成。拟南芥PhyB/C/D/E为此类。
    \end{enumerate}
    \item 光受体的一般作用方式:进入细胞核,直接抑制泛素连接酶COP1,或者压制转录抑制因子如PIFs等,解除对光响应基因的抑制。在暗中,光敏色素定位于细胞质中。在接受光信号后,光敏色素进入细胞核,与泛素连接酶COP1 或PIFs转录因子互作,调节基因表达。
    \item 参与调节的光感知系统除红光远红光受体光敏色素外,还有:蓝光紫外光A受体、昼夜节律感知系统等。
    \item 红光受体光敏色素B和蓝光紫外光A受体向光素也是植物热受体。
    \item \textbf{成花素}:一种可以在器官间运输介导开花的物质。光周期信号从\uline{叶片}传递到\uline{茎尖分生组织},并可以在嫁接的植株间传递。
    \item 成花素是\uline{FT蛋白质},可以在韧皮部长距离运输。
    \item 影响光周期信号传递的其他因素:
    \begin{enumerate}
        \item 说明雌性激素与植物成花诱导过程密切相关;
        \item 植物激素可促进或抑制成花,在不同植物中的效应不同。
        \item 长日植物或日中性植物,其营养中\uline{碳氮比升高则开花},反之,则延迟或不开花。
        \item 温度改变可促进或抑制成花,在不同植物中的效应不同。
    \end{enumerate}
\end{enumerate}

\section{植物光周期理论在农业生产上的应用}
\begin{enumerate}
    \item 地理分布与光周期特性:
    \begin{enumerate}
        \item 低纬度,一般分布短日植物;
        \item 高纬度,多分布长日植物;
        \item 中纬度,则长短日植物共存。
        \item 同一纬度,长日植物多在日照较长的春末和夏季开花,如小麦;短日植物如菊花等则多在日照较短的秋季开花。
    \end{enumerate}
    \item \textcolor{red}{引种育种的选择:}
    \begin{table}[h]
        \centering
        \begin{tabular}{ccc}
            \toprule
            &\textbf{短日植物}&\textbf{长日植物}\\
            \midrule
            \textbf{从北方引种到南方}&晚熟品种&早熟品种\\
            \textbf{从南方引种到北方}&早熟品种&晚熟品种\\
            \bottomrule
        \end{tabular}
    \end{table}
    \item 调节营养生长和生殖生长:
    \begin{enumerate}
        \item 对以收获营养体为主的作物,可通过控制光周期来抑制其开花;
        \item 在花卉栽培中,已经广泛地利用人工控制光周期的办法来提前或推迟花卉植物开花。
    \end{enumerate}
\end{enumerate}