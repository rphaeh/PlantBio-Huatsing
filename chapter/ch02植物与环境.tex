\chapter{植物与环境}
\section{植物与病原菌}
\begin{enumerate}
    \item \textcolor{red}{\textbf{Koch法则}:将一种微生物确定为某种疾病的病原体的标准}
    \begin{enumerate}
        \item 每个病例中,在特定部位能找到特定的病原菌,而正常生物体中同样的部位没有;
        \item 怀疑的病原菌能够被分离并能获得纯培养;
        \item 怀疑的菌接种到同种或同类动物上,能引起相同或相似的症状;
        \item 从发病的动物上能再次分离到这个菌。
    \end{enumerate}
    \item Koch法则的例外:
    \begin{enumerate}
        \item 有些病原微生物无法获得纯培养,如病毒、衣原体;
        \item 多种病原菌引起一种疾病。
    \end{enumerate}
    \item 病原菌导致植物患病的例子:
    \begin{enumerate}
        \item \textbf{致病疫霉}导致\textbf{马铃薯晚疫病};
        \item \textbf{支链淀粉欧文氏菌}导致梨和苹果的枯萎病;
        \item \textbf{烟草花叶病毒}导致烟草患病的病原微生物。
    \end{enumerate}
    \item 病原体致病的三要素:
    \begin{enumerate}
        \item \textbf{宿主}具有易感性;
        \item \textbf{病原体}能攻破宿主的防卫系统;
        \item \textbf{环境}有利于病原体的作用。
    \end{enumerate}
    \item 植物病原体的分类:
    \begin{enumerate}
        \item \textbf{Necrotroph}:杀死细胞,然后消耗细胞组分;
        \item \textbf{Biotroph}:寄生于宿主但不杀死宿主细胞;
        \item \textbf{Hemibiotroph}:可在Necrotroph和Biotroph之间切换。
    \end{enumerate}
    \item \textbf{水杨酸酯}是\uline{促进防卫反应}的信号,\textbf{茉莉酸酯}是\uline{抑制防卫反应}的信号。植物对病毒的防卫反应由\textbf{siRNA}介导。
    \item 细胞壁在植物防卫病原菌中的作用:
    \begin{enumerate}
        \item 完整的细胞壁是植物抵御微生物侵染的物理屏障;
        \item 被侵染的细胞壁迅速木质化形成死细胞层。
    \end{enumerate}
\end{enumerate}

\section{植物与其他生物}
\begin{enumerate}
    \item 节肢动物损害植物体导致植物减产。植物表面的\textbf{毛状体}产生特殊化学物质防止节肢动物侵害。
    \item 植物产生\uline{毒素}、\uline{可转化为毒素的物质}或\uline{营养抑制剂}来抵御动物捕食。一些植食动物已经进化出对植物毒素的耐受性。
    \item 草食动物引起的植物挥发物也可吸引寄生性节肢动物。
    \item 一些植物通过吸引特定的动物成为“驻地保镖”来防止其他动物的侵害。
    \item 花和传粉者可发生共同进化。植物共生促进繁殖和养分吸收。一些花具有复杂的形状和图案,以促进授粉。
    \item 植物与根共生体形成相互联系。豆科植物与根瘤菌互利共生。
\end{enumerate}