\chapter{植物的矿质营养}
\section{植物体内的必需元素}
\begin{enumerate}
    \item 将植物组织在100-105度下烘干,留下干物质(占5-90\%)。将干物质在600度下灼烧,部分元素(C、H、O、N和部分S)在燃烧时挥发。灼烧后剩下的固体残留物称为灰分,灰分中含有的元素称为\textbf{灰分元素},也叫\textbf{矿质元素},包括小部分S、全部P(非金属元素,以矿物质的硫酸盐、磷酸盐形式存在)和所有金属元素。\uline{把N也归入矿质元素中。}
    \item 不同植物灰分含量不同:盐生植物>中生植物>水生植物。
    \item 不同组织器官灰分含量不同:叶>根茎>种子>木质部。
    \item 植物必需元素的确定方法:\uline{不可缺少性}、\uline{不可替代性}、\uline{直接功能性}。
    \item 17种植物必需元素:
    \begin{enumerate}
        \item 9种大量元素:C、H、O、N、P、K、Ca、Mg、S;
        \item 8种微量元素:Cl、Fe、B、Mn、Zn、Cu、Ni、Mo。
    \end{enumerate}
    \item \textcolor{red}{植物主要必需元素的生理功能:}
    \begin{enumerate}
        \item N:组成细胞质、细胞核、细胞壁,是构成生命体各类化合物的元素。N过多,叶色深绿,营养体徒长,抗逆能力差。N过少,植株小,叶色淡,籽粒不饱满,产量低。
        \item P:四大生命分子代谢。缺磷导致植物生长受阻,且由于糖分运输受阻,叶片中积累大量糖分,易形成花色素苷,导致叶色暗绿。
        \item K:调节细胞渗透势,提高原生质水合程度;促进物质合成和代谢,作为60多种酶的泛化剂;促进能量代谢;促进物质运输;与淀粉和纤维素的形成有关,抗倒伏。
        \item Ca:细胞的重要结构成分,组成细胞壁的果胶钙,是磷脂与蛋白质间的桥梁;细胞内的第二信使;酶的活化剂;与抗病有关,促进受伤部位形成愈伤组织;结合草酸以消除毒害。
        \item Fe:呼吸链和光合链中酶的辅基;叶绿素合成需要;固氮酶的成分。(吸收形式为Fe$_2$O$_3$。)
        \item Zn:色氨酸合成酶的必要成分;叶绿素合成需要。
        \item Ni:脲酶的必需组分。
    \end{enumerate}
\end{enumerate}

\section{植物对矿质元素的吸收及运输}
\begin{enumerate}
    \item 水和营养离子吸收部位:根顶端部分,特别是\uline{根毛区(成熟区)}。
    \item 根系吸收矿质元素的过程和途径:
    \begin{enumerate}
        \item 土壤溶液中多数以离子形式存在的矿质元素先通过\uline{离子交换}或\uline{接触交换}被吸附在根组织表面;
        \item 自由扩散进入根内部自由空间;
        \item 经质外体或共质体途径到达内皮层;
        \item 经共质体途径穿越内皮层细胞,进入根组织维管束的木质部导管。然后,随木质部汁液在蒸腾拉力和根压的共同作用下上运至植物的地上部分。
    \end{enumerate}
    \item 根系吸收矿质营养与吸收水分的关系:相互联系,相对独立。
    \begin{itemize}
        \item 相互联系:
        \begin{enumerate}
            \item 矿质营养元素只有溶于水才能被植物吸收;
            \item 活细胞对矿质元素的吸收导致细胞水势降低,促进细胞吸水。
        \end{enumerate}
        \item 相对独立:
        \begin{enumerate}
            \item 水分子和矿质元素通过不同的跨膜转运蛋白进行跨膜吸收;
            \item 细胞对水分和矿质元素的吸收不成比例;
            \item 动力:水分吸收动力是\uline{蒸腾拉力},矿质元素的吸收动力是\uline{代谢能量};
            \item 运输去向:水分运输至叶片,矿质元素运输到生长中心。
        \end{enumerate}
    \end{itemize}
    \item 根系对离子吸收具有选择性:
    \begin{enumerate}
        \item \textbf{生理酸性盐}:如硫酸铵,铵离子吸收大于硫酸根离子,伴随根细胞释放氢离子,使环境酸化。
        \item \textbf{生理碱性盐}:如硝酸钠或硝酸钙,硝酸根离子吸收大于阳离子,并伴随氢离子吸收,使环境碱化。
        \item \textbf{生理中性盐}:如硝酸铵,阴阳离子平衡吸收,不会改变土壤pH。
    \end{enumerate}
    \item \textcolor{red}{\textbf{单盐毒害}}:若将植物培养在某一单盐溶液中,即使浓度很低(低于正常情况下受害浓度),不久即呈现不正常状态,最后枯死。原因是当培养在仅含有一种盐类溶液中的植物,将很快的积累离子。
    \item \textcolor{red}{\textbf{离子拮抗}}:若在单盐溶液中加入少量其他盐类,单盐毒害即可消除。
    \item 要使植物生长良好,必须使其在含有适当比例的多盐溶液中生长,这种能使植物正常生长的混合溶液称为\textbf{平衡溶液}。对于海生植物来说,海水就是其平衡溶液,陆生植物,土壤溶液一般就是其平衡溶液。
    \item 影响根系吸收矿质元素的主要因素:
    \begin{enumerate}
        \item 土壤温度:存在最适温度;
        \item 通气状况:O$_2$较多时吸收旺盛,CO$_2$过高时抑制吸收;
        \item 土壤溶液浓度:存在饱和吸收速率;
        \item 土壤pH:酸性环境有得于阴离子吸收,碱性环境有利于阳离子吸收,但极端pH都不利于矿质吸收。
    \end{enumerate}
    \item 除根以外,植物地上部分也可以吸收矿质营养,这一过程称为\textbf{根外营养}。地上部分吸收矿物质的器官主要是叶片,所以也称为\textbf{叶片营养}。通过\uline{气孔、皮孔、角质层}吸收。
    \begin{itemize}
        \item \textbf{角质层}:多糖和角质的混合物,其上有裂缝,呈细微的孔道。
        \item \textbf{外连丝}:贯穿植物表皮细胞外侧壁的一种胞间连丝样纤细通道,可能起源于胞间连丝。从胞内或质膜延伸到胞壁表面,可能是细胞内、外间物质交流的途径之一。
        \item 根外吸收途径:溶液$\to$角质层孔道$\to$表皮细胞外壁$\to$外连丝$\to$表皮细胞的质膜$\to$细胞内部。
    \end{itemize}
    \item 矿质元素在体内的运输:
    \begin{itemize}
        \item 形式:N为有机物,其他元素为无机盐;
        \item 途径:根吸收的矿质元素由\uline{木质部导管}向上运输、横向运输;叶片吸收的矿质元素由\uline{韧皮部}向下运输、横向运输。
    \end{itemize}
    \item 矿质元素的循环利用:
    \begin{enumerate}
        \item 可再利用元素:在植物体内可反复多次的被利用,如N、P、K、Mg、Cl;
        \item 不可再利用元素:在植物体内形成稳定的化合物,不易移动,不易被循环利用,如Fe、S、Ca、Mn、B。
    \end{enumerate}
\end{enumerate}

\section{植物细胞跨膜离子运输机制}
\begin{enumerate}
    \item 离子跨膜运输蛋白:离子通道、离子载体、离子泵。
    \item 离子载体运输的物质:矿质营养元素离子、呈离子状态的有机代谢物(例如一些氨基酸、有机酸)。
    \item 植物细胞膜上确认的离子泵:
    \begin{enumerate}
        \item 质膜上的H$^+$-ATP酶和Ca$^{2+}$-ATP酶;
        \item 液泡膜上的H$^+$-ATP酶和Ca$^{2+}$-ATP酶;
        \item 内膜系统上的$H^+$-焦磷酸酶。
    \end{enumerate}
    \item \textbf{初级主动运输}:植物细胞膜上由 H+-ATP 酶所执行的主动运输过程。
    \item \textbf{次级主动运输}:由 H+-ATP 酶活动所建立的跨膜质子电化学梯度所驱动的其他无机离子或小分子有机物质的跨膜运输过程。次级主动运输是协同运输过程,即两种离子同时被跨膜运输的过程,分为同向和反向运输两个不同的过程。要把某种有害的离子(如Na$^+$)逆浓度梯度主动排出细胞,是反向共运输。
\end{enumerate}