\chapter{植物激素}
\begin{enumerate}
    \item \textbf{植物生长物质}的概念:植物生长物质(plant growth substances)是指一些小分子化合物,它们在极低的浓度下便可以显著地影响植物的生理功能,包括天然存在的\uline{植物内源激素}和人工合成的\uline{植物生长调节剂}。
    \item 目前确定的9种植物激素:生长素、赤霉素、细胞分裂素、脱落酸、乙烯、油菜素内酯、茉莉酸(酯)、水杨酸(酯)、独脚金内酯。
    \item 其他植物生长物质:多胺、系统素、寡糖素、小分子多肽……
\end{enumerate}

\section{生长素}
\subsection{生长素的发现和种类}
\begin{enumerate}
    \item 生长素的发现:1880年达尔文(Darwin)父子利用燕麦胚芽鞘进行向光性实验,发现在单侧光照射下,胚芽鞘向光弯曲;如果切去胚芽鞘的尖端或在尖端套以锡箔小帽,单侧光照便不会使胚芽鞘向光弯曲;如果单侧光线只照射胚芽鞘尖端而不照射胚芽鞘下部,胚芽鞘还是会向光弯曲。
    \item 天然存在的生长素:吲哚乙酸(IAA)、4-氯吲哚乙酸、苯乙酸、吲哚丁酸。
    \item 化学合成的生长素:2,4-二氯苯氧乙酸、α-萘乙酸、2-甲基-4-氯苯氧乙酸、β-萘乙酸。
    \item 抗生长素类物质:2,3,5-三碘苯甲酸、α-(β-氯苯氧)异丁酸、2,4,6-三氯苯氧乙酸、萘基邻氨甲酰苯甲酸。
\end{enumerate}
\subsection{生长素的合成}
\begin{enumerate}
    \item 合成部位:\uline{快速分裂的组织},例如\uline{茎尖分生组织},\uline{幼嫩叶片},\uline{发育中的种子和果实};\uline{成熟叶片}、\uline{根尖}也可以少量合成。
    \item 合成前体:\uline{色氨酸}或者\uline{吲哚}。
    \item 合成途径:色氨酸依赖途径、非色氨酸依赖途径。
    \begin{enumerate}
        \item \textbf{色氨酸依赖途径}:
        \begin{enumerate}
            \item \textbf{吲哚-3-丙酮酸途径}:\uline{转氨}生成吲哚-3-丙酮酸,再经过\uline{脱羧}形成吲哚-3-乙醛,\uline{氧化}形成吲哚-3-乙酸。
            \item \textbf{色胺途径}:\uline{脱羧}生成色胺,再经过\uline{转氨}形成吲哚-3-乙醛,\uline{氧化}形成吲哚-3-乙酸。
            \item \textbf{吲哚乙腈途径}:生成吲哚-3-乙腈后发生\uline{水解}生成吲哚-3-乙酸。
            \item 吲哚-3-乙酰胺途径:生成吲哚乙酰胺后\uline{水解}生成吲哚-3-乙酸。
        \end{enumerate}
        \item \textbf{非色氨酸依赖途径}:在色氨酸不参与的情况下,\uline{由吲哚-3-甘油磷酸酯(IGA)}、\uline{吲哚-3-乙腈(IAN)}或\uline{吲哚-3-丙酮酸(IPA)}直接形成IAA。
    \end{enumerate}    
\end{enumerate}
\subsection{生长素的运输}
\begin{enumerate}
    \item \textbf{极性运输}:茎尖、根尖产生的生长素只能从形态学的\uline{上端向下端}运输。不受重力方向的影响,沿着植株茎干和根系形成一个\uline{两头多中间少}的生长素浓度梯度。生长素是\uline{唯一}具有极性运输性质的植物激素。
    \item 极性运输产生的生长素浓度梯度影响植物的许多生理过程,如茎的伸长生长、顶端优势、不定根发生以及叶片脱落等。
    \item 即使将竹子切段倒置,根也会从其形态学基部长出来,在基部形成根的主要原因是茎中生长素的极性运输,与重力无关。
    \item 在不同组织中可能运输的细胞不同:
    \begin{enumerate}
        \item 燕麦胚芽鞘中是\uline{非维管束组织细胞};
        \item 双子叶植物茎中,主要是\uline{维管束薄壁细胞}。
    \end{enumerate}
    \item 非极性运输:\uline{成熟叶片}合成的生长素随生长素结合物经\uline{韧皮部}的长距离运输。作用于\uline{形成层}或\uline{侧根发生}。
\end{enumerate}
\subsection{生长素的代谢}
\begin{enumerate}
    \item 两种状态的IAA:
    \begin{enumerate}
        \item 游离态:具生理活性,极性运输;
        \item 结合态:非/低活性,与葡萄糖、氨基酸、肌醇等结合,为贮藏或运输形式,非极性运输。
    \end{enumerate}
    \item IAA的降解途径:
    \begin{enumerate}
        \item 酶解:氧化成羟吲哚-3-乙酸,或在过氧化物酶作用下脱羧成亚甲基氧代吲哚。
        \item 光解:过程和生理意义不清楚。
    \end{enumerate}
    \item 生长素的合成和降解代谢在\uline{细胞质}中进行。
    \item 生长素的储存:IAA在植物细胞内的存在部位主要是\uline{细胞质}和\uline{叶绿体},结合态IAA则存在于\uline{细胞质}中。
    \item IAA在\uline{细胞质}中发生从头生物合成、结合物合成以及非脱羧降解反应等;\uline{叶绿体}中的生长素受到\uline{保护},不受代谢的影响,但其浓度受细胞质中IAA浓度的平衡调节。
    \item 细胞中IAA水平的稳态:
    \begin{enumerate}
        \item 来源:色氨酸途径、非色氨酸途径;
        \item 去路:运输、氧化脱羧降解;
        \item 稳态的维持:结合、分室化。
    \end{enumerate}
\end{enumerate}
\subsection{生长素的生理效应}
\begin{enumerate}
    \item \textcolor{red}{生长素的生理效应:}
    \begin{enumerate}
        \item 促进细胞伸长生长;
        \item 诱导维管束分化;
        \item 促进侧根和不定根发生;
        \item 影响花及果实发育;
        \item 引起顶端优势;
        \item 促进叶片的扩大、光合产物的运输等;
        \item 延迟花的脱落和叶子脱落;
        \item 引起向光性,向地性。
    \end{enumerate}
    \item 促进细胞伸长生长:酸生长理论、细胞壁延展性和细胞壁延展蛋白。
    \item 生长素的一个非常重要的生理作用就是诱导和促进植物细胞的分化,尤其是促进植物维管组织的分化。黄瓜茎组织中IAA诱导的伤口周围木质部的再生。
    \item 促进侧根和不定根发生:首先是生长素极性运输到中柱鞘细胞并在其中积累到一定浓度,诱导这些细胞开始分裂;然后生长素刺激并维持细胞持续分裂、生长和分化,最后形成侧根。农业生产上用作促进插条不定根的形成。
    \item 烟草愈伤组织器官分化:\uline{IAA与CTK的比值低时,有利于芽的形成},而抑制根的分化;反之, IAA与CTK的比值高时,则有利于根的形成,而抑制芽的分化。CTK/IAA比值为1,只有愈伤组织生长而\uline{无分化}。
    \item 影响花及果实发育
    \begin{itemize}
        \item 许多植物上,外源施用生长素\uline{抑制}花的形成,这种抑制作用可能是一种衍生的次级反应,即可能是生长素诱导的\uline{乙烯}产生的抑制作用。例外:生长素或乙烯处理可以强烈\uline{促进凤梨属植物开花}。
        \item 花芽一旦形成,生长素在花的发育以及性别决定方面中开始担负着重要的角色。发育早期利用生长素处理可以使其发育成为\uline{雌性}花。
        \item 1950年Nitsch发现,种子是\uline{促进果实膨大}的生长素供应源。
        \item 生长素可用于\uline{促进早期果实脱落},\uline{防止未成熟果实的脱落}。
    \end{itemize}
\end{enumerate}

\section{赤霉素}
\subsection{赤霉素的发现和种类}
\begin{enumerate}
    \item 赤霉素的发现:1926年,日本黑泽英发现水稻植株徒长的恶苗病(不结子);1935年,日本薮田贞次从水稻恶苗病菌中分离得到一种与水稻徒长有关的物质,并定名为赤霉素;1938年,薮田和住木从赤霉菌培养基的滤液中分离出了两种结晶,定名为赤霉素A、赤霉素B。    
    \item 结构:赤霉素是一类\uline{双萜酸}化合物,由\uline{4个异戊二烯单位}组成。基本结构是\uline{赤霉烷}。由于赤霉素都含有羧基, 所以赤霉素呈酸性。
    \item 分类:赤霉素包含19或20个碳原子,据此可将赤霉素分为C$_{19}$GA和C$_{20}$GA两类。C$_{19}$GA类比C$_{20}$GA类的相对生理活性高。
\end{enumerate}
\subsection{赤霉素的合成和代谢}
\begin{enumerate}
    \item 赤霉素合成:质体内质网和细胞质的协同作用,共三个步骤:
    \begin{enumerate}
        \item 质体中环化反应生成\uline{贝壳杉烯};
        \item 内质网中连续氧化反应生成GA12醛/GA12;
        \item \uline{细胞质}中由\uline{GA12醛}形成所有GA$_x$,有两条合成支路——\uline{早期C13非羟化途径}、\uline{早期C13羟化途径},催化酶为\uline{P450单加氧酶}或\uline{非血红素双加氧酶},抑制剂为\uline{环己烷三酮}。
    \end{enumerate}
    \item 用于赤霉素合成的基本的异戊二烯单元是\uline{异戊烯二磷酸(IPP)}。绿色组织中GA合成所用的IPP是由三磷酸甘油醛和丙酮酸在质体合成的。但在有些非绿色组织中,比如富含GA的南瓜胚乳中,IPP是利用甲瓦龙酸在细胞质中合成的。因此, 不同组织中用于GA合成的IPP可能来自不同的细胞器。
    \item 赤霉素的合成部位:合成最活跃的器官是幼芽、幼叶和上部茎节以及发育中的种子和果实。成熟叶片的叶肉细胞\uline{不能进行贝壳杉烯的合成},但可以进行赤霉素第三步骤的合成,前体是\uline{从茎的分生组织转移而来},在叶片中被转化为活性赤霉素。尚没有确切证据说明根系可以合成赤霉素。\uline{除了根以外,合成部位与IAA类似。}
    \item 赤霉素的运输:茎中合成的赤霉素可以通过\uline{韧皮部}运输到植株的其他部分。
    \item 赤霉素的代谢:
    \begin{enumerate}
        \item \uline{GA的失活}主要是通过\uline{2β-羟化反应}(2位碳原子上的羟化反应),使活性赤霉素以及活性赤霉素前体不可逆地失去活性。 
        \item \uline{GA结合态失活}:GA羧基和羟基可分别与单糖形成糖脂或糖苷,这些GA结合物有储存和运输的功能。
    \end{enumerate}
\end{enumerate}
\subsection{赤霉素的生理效应}
\begin{enumerate}
    \item \textcolor{red}{赤霉素的生理效应:}
    \begin{enumerate}
        \item 促进茎的伸长;
        \item 调节植物幼态和成熟态之间的转换;
        \item 影响花芽分化和性别控制诱导开花;
        \item 打破休眠、促进萌发。
    \end{enumerate}
    \item 促进茎的伸长:
    \begin{enumerate}
        \item 促进整株植物生长:
        \begin{itemize}
            \item 用GA处理,能显著促进植株茎的伸长生长,对矮生突变品种的效果特别明显。
            \item GA对离体茎切段的伸长没有明显的促进作用,而IAA对整株植物的生长影响较小,却对离体茎切段的伸长有明显的促进作用。
            \item GA 施用导致短日照下植物茎的伸长及花的发育。
            \item 许多植物的茎的伸长与开花需要通过低温处理,GA 可代替此作用。
        \end{itemize}
        \item 促进节间的伸长:
        \begin{itemize}
            \item GA主要作用于已有节间伸长,不促进节数的增加。导致茎干直径减小、叶片变小、叶色变浅。
            \item 不存在超最适浓度的抑制作用。
            \item 玉米实验中,赤霉素促进了矮生突变体茎干的明显伸长,但是对野生型的植株却没有或仅有很小的效果。
        \end{itemize}
    \end{enumerate}
    \item GA合成途径基因突变影响到GA的合成,降低了有活性的GA (GA1) 的浓度,使植株矮化。因此GA基因被称为“\textbf{绿色革命基因}”。
    \item 调节植物幼态和成熟态之间的转换:据植物种类的不同,外源施用赤霉素可以控制植物在幼态和成熟态之间的转变。比如,GA3可以诱导常春藤从成熟态转变为幼态;GA4+GA7处理可以诱导许多幼态的针叶植物进入成熟态。
    \item 影响花芽分化和性别:
    \begin{enumerate}
        \item 单子叶:外源赤霉素对玉米花性别决定的主要作用是\uline{抑制雄花}发育。
        \item 双子叶:外源赤霉素会\uline{促进雄花}的形成,而赤霉素生物合成抑制剂可以促进雌花的发育。
        \item 对于苹果,外源赤霉素具有\uline{促进坐果}的作用。
    \end{enumerate}
    \item 打破休眠,促进萌发:
    \begin{itemize}
        \item GA可以\uline{刺激胚芽的生长},可以\uline{诱导水解酶的合成},分解种子储存的营养物质。
        \item GA可\uline{代替光照和低温打破休眠}。对于需光和需低温萌发的种子(如莴苣、紫苏、李和苹果等的种子), GA可以催化种子内贮藏物质的降解,供胚的生长发育所需,打破休眠。
        \item 在正在萌发的谷类种子中,GA处理可\uline{诱导糊粉层α-淀粉酶、蛋白酶产生}。这个性质可以用在酿造工艺中,GA\uline{加速酿造时的糖化过程},并\uline{降低萌芽的呼吸消耗},从而降低酿酒成本。 
    \end{itemize}
    \item 与IAA作用比较:
    \begin{enumerate}
        \item GA特有:促进整株植物生长,打破休眠、促进种子萌发,促进某些二年生植物抽苔开花,诱导花性别分化。
        \item IAA特有:促进愈伤组织的形成,极性运输,顶端优势、抑制腋芽。
        \item 共有:促进细胞分裂,促进细胞伸长,促进酶活性,诱导无籽果实。
    \end{enumerate}
\end{enumerate}

\section{细胞分裂素}
\subsection{细胞分裂素的发现}
\begin{enumerate}
    \item 1940s,Skoog et al:培养基+椰子乳或酵母提取液,促进细胞分裂。
    \item 1955年,米勒和斯库格等偶然将存放了4年的鲱鱼精细胞DNA加入到烟草髓组织的培养基中,发现能诱导细胞的分裂,但用新提取的DNA却无促进细胞分裂的活性,如将其在pH<4的条件下进行高压灭菌处理,则又可表现出促进细胞分裂的活性。
    \item 1956年,米勒等从高压灭菌处理的鲱鱼精细胞DNA分解产物中纯化出了激动素结晶,并鉴定出其化学结构为6-呋喃氨基嘌呤,分子式为C$_{10}$H$_9$N$_{50}$,分子量为215.2,接着又人工合成了这种物质,命名为激动素(kinetin, KT) 。
    \item 1963年,莱撒姆从未成熟的玉米籽粒中分离出了一种类似于激动素的细胞分裂促进物质,命名为玉米素(zeatin,ZT)。
    \item 1964年确定其化学结构为6-(4-羟基-3-甲基-反式-2-丁烯基氨基)嘌呤,分子式为C$_{10}$H$_{13}$N$_{5O}$,分子量为129.7。
    \item 1965年,斯库格等提议将来源于植物的、其生理活性类似于激动素的化合物统称为细胞分裂素(cytokinin, CTK,CK)。
\end{enumerate}
\subsection{细胞分裂素的结构和种类}
\begin{enumerate}
    \item 细胞分裂素的结构:腺嘌呤的衍生物。
    \item 四种天然细胞分裂素:反式-玉米素、顺式-玉米素、双氢玉米素、异戊烯基腺嘌呤。
    \item 合成的细胞分裂素:芐基腺嘌呤、四氢吡喃芐基腺嘌呤。
    \item 细胞分裂素拮抗剂:二苯脲。
\end{enumerate}
\subsection{细胞分裂素的合成、运输与代谢}
\begin{enumerate}
    \item 细胞分裂素(CTK)的生物合成:
    \begin{enumerate}
        \item 从头合成——主要合成途径:从\uline{异戊烯基焦磷酸(IPP,与GA合成前体相同)}出发,得到\uline{异戊烯基腺苷磷酸(iPMP)},再合成\uline{异戊烯腺苷},并衍生得到细胞分裂素。
        \item tRNA的降解合成途径(不能满足细胞要求):酵母tRNA降解时释放出细胞分裂素,或异戊烯腺苷。        
    \end{enumerate}
    \item 细胞分裂素合成部位:根尖分生组织、胚珠、韧皮部、叶腋、幼穗、幼果、发育中的叶片(\uline{与IAA类似})。
    \item 细胞分裂素的运输途径:根中合成的CK随\uline{导管}运输到地上部,地上部器官合成的CTK可能不外运。
    \item 细胞分裂素的外运形式:\uline{玉米素核苷}。
    \item 细胞分裂素的代谢:
    \begin{enumerate}
        \item CTK结合物的合成和水解:
        \begin{itemize}
            \item 玉米素、二氢玉米素和异戊烯基腺嘌呤是植物中天然存在的游离态的细胞分裂素,是CTK的活性形式。
            \item 异戊烯基腺苷(iPA)、甲硫基异戊烯基腺苷等通过糖基化、乙/丙酰基化与葡萄糖、核苷酸或氨基酸共价结合形成结合态的细胞分裂素(核苷、葡糖苷以及乙/丙酰化CTK)。结合态CTK是非活性形式。
        \end{itemize}
        \item 细胞分裂素的氧化降解:由\uline{细胞分裂素氧化酶}催化,生成腺嘌呤和3-甲基丁醛。
    \end{enumerate}
\end{enumerate}
\subsection{细胞分裂素的生理效应}
\begin{enumerate}
    \item 细胞分裂素被定义为具有与反式玉米素具有相同生物活性的物质。
    \item \textcolor{red}{细胞分裂素的生理效应:}
    \begin{enumerate}
        \item 促进细胞分裂和膨大;
        \item 促进芽和不定芽生长分化;
        \item 促进侧芽发育、消除顶端优势;
        \item 延缓叶片衰老。
    \end{enumerate}
    \item 促进双子叶植物幼苗子叶的扩大:IAA和CTK都是通过\uline{激活周期依赖性蛋白激酶(CDK)}活性来调节细胞周期的。
    \item 促进侧芽和不定芽的生长分化。
    \item 促进侧芽发育、消除顶端优势:细胞分裂素具有诱导侧芽生长的生理效应,是削弱或打破顶端优势的激素。
    \item 延缓叶片衰老:幼嫩叶片可以产生CTK,成熟叶片不能合成,主要是由根系合成并运输而来。CTK处理可延缓叶片衰老。
\end{enumerate}

\section{脱落酸}
\subsection{脱落酸的发现}
\begin{enumerate}
    \item 1963年,Addicott等(美)研究棉花幼果脱落,提纯、结晶一种物质,命名为脱落素II(abscisin II)。
    \item Wareing等(英)桦树促进芽休眠的物质:休眠素。
    \item 1965年确定了化学结构式。
    \item 1967年第六届国际植物生长物质会议上统一定名为脱落酸 (absiaisic acid,ABA)。
\end{enumerate}
\subsection{脱落酸的结构与活性}
\begin{enumerate}
    \item 生物合成的ABA构型是S对映体(右旋),人工合成的ABA是S和R对映体的消旋混合物。
    \item 存在部位:\uline{从根尖到茎尖的所有部位}。
    \item 合成部位:\uline{几乎所有含叶绿体或质体的细胞}。
\end{enumerate}
\subsection{脱落酸的生物合成、运输和代谢}
\begin{enumerate}
    \item ABA的生物合成开始于\uline{叶绿体/质体}内,在其中完成大部分过程,结束于\uline{细胞质}中。
    \item 脱落酸生物合成的途径:
    \begin{enumerate}
        \item 类萜途径(直接途径,C15途径):从\uline{甲瓦龙酸(MVA)}出发(与GA相同)。
        \item 类胡萝卜素途径(间接途径,主要途径,C40途径):从\uline{类胡萝卜素}出发。
    \end{enumerate}
    \item \uline{黄质醛}进入细胞质,氧化为\uline{脱落醛},再氧化为脱落酸。
    \item 植物激素GA、CTK、ABA、 BR都是类萜化合物,它们的合成起始物是相同的——\uline{异戊烯二磷酸(IPP)}。
    \item 脱落酸的代谢:植物体内特定器官内ABA水平的变化会随着环境条件的变化发生大幅度的波动。例如,在\uline{水分胁迫}的条件下,叶片中的ABA水平可以在4-8小时内上升50-100倍,随着水分条件的改善,ABA水平又会在相同的时间内下降到正常水平。
    \item 水分胁迫条件下的叶片中ABA水平的上升是由下列几个原因共同作用:
    \begin{enumerate}
        \item 叶片内ABA生物合成的增加;
        \item 结合态ABA降解释放为自由态;
        \item 叶肉细胞内储存ABA的释放;
        \item 根系内合成的ABA通过蒸腾流向叶片的运输。 
    \end{enumerate}
    \item ABA的失活代谢包括\uline{氧化降解}和\uline{形成结合物}两个途径。
    \item 脱落酸的运输:既可以在木质部运输,也可以在韧皮部运输。
    \begin{itemize}
        \item 用环割的方法破坏韧皮部,会抑制ABA向根系的运输和积累,所以叶片内的ABA运输主要依赖韧皮部运输。
        \item 土壤水分状况良好时,向日葵植株的木质部液内的ABA浓度大约是1.0-15.0nM,土壤水分胁迫时,ABA浓度可以上升到3.0μM。ABA是作为一种干旱信号传输到地上部叶片,通过\uline{诱导气孔关闭}来降低叶片水分蒸腾。
    \end{itemize}
\end{enumerate}
\subsection{脱落酸的生理功能}
\begin{enumerate}
    \item \textcolor{red}{脱落酸的生理功能:}
    \begin{enumerate}
        \item 调节种子发育和休眠;
        \item 抑制幼苗生长;
        \item ABA是逆境信号;
        \item 促进叶片衰老。
    \end{enumerate}
    \item 调节种子发育和休眠:
    \begin{enumerate}
        \item ABA促进了胚胎耐干燥性的形成。干燥脱水会严重破坏细胞膜及细胞器的结构。
        \item ABA促进储藏物质积累。ABA是储藏蛋白合成必要的控制因子。
        \item ABA/GA比例高有利于种子休眠。
        \item ABA可抑制\textbf{胎萌现象},即种子在未脱离母体植株前就开始萌发的现象。
    \end{enumerate}
    \item 逆境信号:一般来说,干旱、寒冷、高温、盐渍和水涝等逆境都能使植物体内ABA迅速增加,同时抗逆性增强。因此,ABA被称为应激激素或胁迫激素。
    \item 调节气孔运动:水分胁迫条件下叶片保卫细胞内ABA水平的升高\uline{促进了气孔的关闭}、减少了蒸腾,维持了叶片的水分平衡。ABA信号\uline{激活细胞质膜和内膜钙通道},引起细胞内钙浓度升高;经过一系列信号转导,激活外向型K$^+$通道和阴离子通道,K$^+$和阴离子外流,引起\uline{保卫细胞水势升高},\uline{水分外流},细胞膨压下降,气孔关闭。
    \item ABA促进了叶片的衰老,增加了乙烯的生成,间接地促进了叶片的脱落。 
\end{enumerate}

\section{乙烯}
\subsection{乙烯的发现}
\begin{enumerate}
    \item 十九世纪,煤气被利用来作路灯照明,街灯下的树落叶要多。
    \item 1901年确定其中的活性物质为乙烯。
    \item 1910年认识到植物组织能产生乙烯。
    \item 1934年确定乙烯为植物的天然产物。
    \item 1959年用气相色谱定量分析乙烯,乙烯的研究才真正活跃起来,被公认为是植物的天然激素。    
\end{enumerate}
\subsection{乙烯生物合成、代谢降解和运输}
\begin{enumerate}
    \item 生物合成前体:\uline{甲硫氨酸}。直接前体:\uline{ACC}。
    \item 合成途径:经\uline{甲硫氨酸循环}形成5'-甲硫基腺苷(MTA)和ACC,后者生成乙烯。
    \item 关键酶:\uline{ACC合成酶}、\uline{ACC氧化酶}。
    \item 合成部位:在植物的\uline{所有活细胞}中都能合成乙烯。形成层和茎节区域是乙烯合成最活跃的部位。
    \item 诱导因素:\uline{叶片脱落}、\uline{花器官衰老}或者\uline{果实成熟}以及\uline{逆境/伤害}因素都会诱导植物体内乙烯的大量合成。
    \item 代谢降解:形成其他化合物或彻底氧化。
    \item 乙烯是气态激素,体内的运输性较差。短距离运输可以通过细胞间隙进行\uline{扩散},扩散距离非常有限。
    \item 乙烯的长距离运输依靠\uline{ACC在木质部溶液中的运输}。ACC是作为一个可以运输的化学信号,是乙烯长途运输的“载体”。
\end{enumerate}
\subsection{乙烯的生理功能}
\begin{enumerate}
    \item \textcolor{red}{乙烯的生理功能:}
    \begin{enumerate}
        \item 抑制生长的三重反应;
        \item 促进果实成熟;
        \item 促进衰老;
        \item 促进离层形成;
        \item 诱导不定根和根毛发生;
        \item 抑制芒果开花,促进菠萝开花;
        \item 促进黄瓜雌花的发育;
        \item 抗逆反应。
    \end{enumerate}
    \item 乙烯在暗环境抑制生长的三重反应:\uline{茎伸长受抑制}、\uline{根生长受抑制}、\uline{上胚轴水平生长(顶端弯钩)}。
    \item \textbf{离层}:植物的叶片、果实和花朵等器官在衰老后都会发生脱落。脱落发生在这些器官基部的一些特殊的细胞层,称为离层。IAA可以抑制脱落的发生,但用过高浓度的IAA或合成的2,4-D反而会诱导乙烯的发生,促进脱落。
\end{enumerate}